\chapter{Notes}

\section{General Notes}

\begin{itemize}
    \item hadron = composite particle composed of quarks
    \item meson = hadron with equal number of quarks and antiquarks
    \item baryon = hadron with unequal number of quarks and antiquarks (usually 3)
          \begin{itemize}
              \item baryon number: $B=(-)\frac{1}{3}$ for (anti-)quarks
          \end{itemize}
\end{itemize}

\section{\cite{BuszaEtAl_2018}}


\paragraph*{Why Heavy Ion Collisions?}\mbox{}\\
\begin{itemize}
    \item study particle production in QCD
    \item QGP = matter of the universe few microseconds after Big Bang where $T>\Lambda_{QCD}\sim(\text{size of hadron})^{-1}$
    \item QGP is a liquid, not made of hadrons but is also not just weakly coupled gas of gluons and quarks
    \item study (strikingly small) ratio $\frac{\eta}{s}\approx\frac{1}{4\pi}$ ($\eta\dots$ shear viscosity, $s\dots$ entropy density) which might have a special meaning in context of AdS/CFT correspondence
    \item fluid dynamic description needs e.o.s. $\hookrightarrow$ for QCD: gained from lattice simulations
    \item How do hydrodynamics emerge?
    \item Experimental realization of $p-p$, $p-Pb$, $d-Au$ Collisions
    \item study phase diagram of QCD: changing collision energy of nuclei changes contribution of baryonic matter of nuclei to matter formed in collisions $\hookrightarrow$ scan phase diagram of non-vanishing baryonic potential $\mu_B\neq 0$
    \item at length scales $<\frac{1}{T}$ QGP is said to be weakly interacting due to asymptotic freedom of QCD, at scales $>\frac{1}{T}$ one observes strongly coupled fluid cells. \todo{Understand notion of weak and strong coupling in QCD at different scales}
\end{itemize}

\paragraph*{Phenomenology}\mbox{}\\
\begin{itemize}
    \item accessible data: nuclei type, energies (to high precision)
    \item inaccessible data: impact parameters, movement of quark constituents
    \item nucleon distribution is well known, quark-gluon distribution in nucleons is well known and almost as in the free nucleon case
    \item some lingo
          \begin{itemize}
              \item participant: nucleon that collides with other nucleon during collision
              \item spectator: nucleon that doesn't collide with anything
              \item binary collision: collision of 2 nucleons
          \end{itemize}
    \item Consider nucleus made of $A$ spheres ("nucleons") and a $p-A$ collision. The probability of the proton hitting any of the $A$ nucleons is $\frac{\sigma_{pp}}{\sigma_{pA}}$. Therefore on average $N_{\text{coll}}=A\frac{\sigma_{pp}}{\sigma_{pA}}$. Experiments: number of produced particles $\sim\# N_{\text{coll}}$
    \item In general $A-A$ experiments: $N_{\text{coll}}$ are determined on-average/within centrality classes via Glauber Model Calculation
\end{itemize}