\chapter{Calculation of Condensed Field}

\section{Relating Fluid and Pion Fields}

The signature is $(-,+,+,+)$.

Start with the Lagrangian of the linear $\sigma$-model for the real-valued $O(4)$-vector $\varphi_a=(\sigma,\mathbf{\pi})$.
\begin{equation}
    \mathscr{L}=-\frac{1}{2}\partial_\mu\sigma\partial^\mu\sigma-\frac{1}{2}\partial_\mu\mathbf{\pi}\partial^\mu\mathbf{\pi}-\frac{1}{2}m^2(\sigma^2+\mathbf{\pi}^2)-\frac{\lambda}{4}(\sigma^2+\mathbf{\pi}^2)^2-\epsilon\sigma
    \label{eq:Lagrangian_LinearSigma1}
\end{equation}
and from \ref{subsubsec:linearsigmamodel} we find, that if $m^2=-\mu^2<0$ SSB occurs (plus, $\epsilon\neq 0$ explicitly breaks the symmetry) with the VEV $v$ and masses of the $\sigma$ and $\mathbf{\pi}$ excitation in $\varphi_a=(v+\delta\sigma,\mathbf{\pi})$ are given by (note the difference in conventions $\lambda\to 6\lambda$ w.r.t. \ref{subsubsec:linearsigmamodel})
\begin{equation}
    \left\{
    \begin{split}
        v_0         & =\frac{\mu}{\sqrt{\lambda}}+\epsilon\frac{1}{2\mu^2}+\mathcal{O}(\epsilon^2) \\
        m_\sigma^2 & =2\mu^2+\mathcal{O}(\epsilon)                                                \\
        m_\pi^2    & =\epsilon\frac{\sqrt{\lambda}}{\mu}+\mathcal{O}(\epsilon^2)
    \end{split}
    \right.\qquad\overset{\text{to lowest order}}{\iff}\qquad
    \left\{
    \begin{split}
        \mu&=\frac{m_\sigma}{\sqrt{2}}\\
        \lambda&=\frac{m_\sigma^2}{2v_0^2}\\
        \epsilon&=v_0m_\pi^2
    \end{split}
    \right.
    \label{eq:LinearSigmaModel_CouplingsMassesRelation}
\end{equation}
Choose now a fixed alignment of the condensate $\mathbf{\pi}=\pi\mathbf{e}$ with $\mathbf{e}\cdot\mathbf{e}=1$ determining the orientation in isospin space. This choice breaks $O(4)$ to $O(2)$ (we restrict ourselves to $SO(2)$) and use to isomorphism $SO(2)\overset{\sim}{=}U(1)$ to write the linear $\sigma$-model as a theory of a complex scalar field $\phi=\frac{1}{\sqrt{2}}(\sigma+\imagu\pi)=\rho e^{\imagu\vartheta}$.
\begin{subequations}
    \begin{align}
        \text{in terms of}\,(\phi,\phi^*)      &  & \mathscr{L} & =-(\partial_\mu\phi)(\partial^\mu\phi^*)+\mu^2\phi\phi^*-\frac{\lambda}{2}(\phi\phi^*)^2+\frac{\epsilon}{\sqrt{2}}(\phi+\phi^*)                                   \\
        \text{in terms of}\,(\rho,\vartheta^*) &  & \mathscr{L} & =-(\partial_\mu\rho)(\partial^\mu\rho)-\rho^2(\partial_\mu\vartheta)(\partial^\mu\vartheta)+\mu^2\rho^2-\frac{\lambda}{2}\rho^4+\sqrt{2}\epsilon\rho\cos\vartheta
    \end{align}
\end{subequations}
The classical equations of motion arising from this are
\begin{subequations}
    \begin{align}
        \text{in terms of}\,(\phi,\phi^*)      &                                                                                                                           & -\Box\phi                                  & =(\partial_t^2-\mathbf{\nabla}^2)\phi=\big[\mu^2-\lambda(\phi^*\phi)\big]\phi+\frac{\epsilon}{\sqrt{2}} \\
        \text{in terms of}\,(\rho,\vartheta^*) &                                                                                                                           &
        -\Box\rho                              & =\big[-(\partial_\mu\vartheta)(\partial^\mu\vartheta)+\mu^2-\lambda\rho^2\big]\rho+\frac{\epsilon}{\sqrt{2}}\cos\vartheta                                                                                                                                                        \\
                                               &                                                                                                                           & -\partial_\mu(\rho^2\partial^\mu\vartheta) & =-\frac{\epsilon}{\sqrt{2}}\rho\sin\vartheta
    \end{align}
\end{subequations}
The conserved Noether current of thse symmetry $\theta\to\theta+\alpha$ (in the limit $\epsilon\to 0$) and energy-momentum tensor are
\begin{subequations}
    \begin{align}
        \alpha j^\mu & =-\frac{\partial\mathscr{L}}{\partial(\partial_\mu\vartheta)}\delta\vartheta=2\alpha\rho^2(\partial^\mu\vartheta)                                                                                                                                                            \\
        T^{\mu\nu}   & =\frac{2}{\sqrt{-g}}\frac{\delta(\sqrt{-g}\mathscr{L})}{\delta g_{\mu\nu}}=2\frac{\partial\mathscr{L}}{\partial g_{\mu\nu}}+g^{\mu\nu}\mathscr{L}=2\big[(\partial^\mu\rho)(\partial^\nu\rho)+\rho^2(\partial^\mu\vartheta)(\partial^\nu\vartheta)\big]+g^{\mu\nu}\mathscr{L}
    \end{align}
\end{subequations}
where $\delta\sqrt{-g}=\frac{-1}{2\sqrt{-g}}\delta g=\frac{-1}{\sqrt{-g}}gg^{\mu\nu}\delta g_{\mu\nu}=\frac{\sqrt{-g}}{2}g^{\mu\nu}\delta g_{\mu\nu}$ was used.

In a tree-level approximation one only needs to solve the classical equations of motion. In the limit $\epsilon\to 0$ a valid solution is
\begin{equation}
    \partial_\mu\vartheta=\const\,,\qquad\rho=\sqrt{\frac{\chi^2+\mu^2}{\lambda}}+\mathcal{O}(\epsilon)\qquad(\chi^2\defeq-(\partial_\mu\vartheta)(\partial^\mu\vartheta))
\end{equation}
We might generalize the solution to the case $\partial_\mu\vartheta\approx\const$ which should be valid in the limit $\chi^2\ll\mu^2=\frac{m_\sigma^2}{2}$. On these solutions one finds
\begin{subequations}
    \begin{align}
        \mathscr{L}\big\vert_{\text{EOM}} & =\rho^2(\chi^2+\mu^2-\frac{\lambda}{2}\rho^2)=\rho^2\frac{\chi^2+\mu^2}{2}                    \\
        T^{\mu\nu}\big\vert_{\text{EOM}}  & =2\rho^2(\partial^\mu\vartheta)(\partial^\nu\vartheta)+g^{\mu\nu}\rho^2\frac{\chi^2+\mu^2}{2}
    \end{align}
\end{subequations}

Assume the dynamics of the field could be described by ideal hydrodynamics, i.e. a conserved current and energy-momentum tensor of the form \todo{I really don't know about the signature}
\begin{subequations}
    \begin{align}
        j^\mu      & =n_s\nu^\mu\qquad(v^\mu v_\mu=-1)          \\
        T^{\mu\nu} & =(\epsilon_s+P_s)v^\mu v^\nu+g^{\mu\nu}P_s
    \end{align}
\end{subequations}
from which the prefactors can be extracted by the contractions
\begin{equation}
    n_s=\sqrt{-j^\mu j_\mu}\,,\qquad\epsilon_s=v_\mu v_\nu T^{\mu\nu}\,,\qquad P_s=\frac{1}{3}(g_{\mu\nu}+v_\mu v_\nu)T^{\mu\nu}
\end{equation}

Identifying the field theoretic with the hydrodynamic viewpoint it immediately follows that
\begin{subequations}
    \begin{gather}
        n_s=2\rho^2\chi\,,\qquad v^\mu=\chi^{-1}(\partial^\mu\vartheta)\qquad(\iff\chi=-v^\mu(\partial_\mu\vartheta))\\
        \epsilon_s=2\rho^2\chi^2-\rho^2\frac{\chi^2+\mu^2}{2}=\rho^2\frac{4\chi^2-(\chi^2+\mu^2)}{2}=\frac{(\chi^2+\mu^2)(3\chi^2-\mu^2)}{2\lambda}
    \end{gather}
\end{subequations}

% \textcolor{red}{\textbf{NOTE THE HANDWRITTEN NOTES SEEM TO HAVE MOSTLY $(+,-,-,-)$ SIGNATURE EXCEPT FOR THE CLAIM $v_\mu v^\mu=-1$, WHIS IS SIMPLY INCONSISTENT}}
% \includepdf{resources/notes_0423.pdf}

\paragraph*{How to apply this?}\mbox{}\\

The freeze-out surface is invariant under rotations (= independent of polar angle $\varphi$) around the collision axis and longitudinal boosts (= independent of rapidity $\eta_s$) and hence parametrized by a one-dimensional curve in the $r\text{-}\tau$-plane. The curve itself may be parametrized by some real parameter $\alpha$, following some mapping $\alpha\mapsto (r(\alpha),\tau(\alpha))$. From the hydro simulation we wish to identify the gradient $\partial_\mu\vartheta\sim u_\mu$ of the complex phase of the condensate field with the fluid $4$-velocity $u_\mu$, hence in order to find the phase of the field, an integration of $\partial_\mu\vartheta$ over the hypersurface is needed and an integration constant $\vartheta_0$ can be chosen freely. Choose $\alpha=\arctan(\tau/r)$ to be the polar angle of the point $(r(\alpha),\tau(\alpha))$ in the $r\text{-}\tau$-plane. Since $r,\tau>0$ $\alpha$ is restricted to the range $[0,\pi]$ and $\vartheta(\alpha)$ on the hypersurface can be calculated via
\begin{equation}
    \vartheta(\alpha)=\vartheta_0+\int_0^\alpha\dt s\frac{\dt\vartheta}{\dt s}=\vartheta_0+\int_0^\alpha\dt s\frac{\partial x^\mu(s)}{\partial s}\partial_\mu\vartheta
    \label{eq:FluidsToFields_complexPhase}
\end{equation}
$\partial x^\mu(s)/\partial s$ represents the tangent vector of the freeze-out surface.

The energy density $\epsilon$ and $4$-velocity $u^\mu$ of the fluid is related to the condensate phase and density via
\begin{subequations}
    \begin{gather}
        -(\partial_\mu\vartheta)(\partial^\mu\vartheta)=\chi^2=\frac{-\mu^2+\sqrt{6\epsilon\lambda+4\mu^4}}{3}\\
        \partial^\mu\vartheta=\chi u^\mu\,,\qquad\rho^2=\sqrt{\frac{\chi^2+\mu^2}{\lambda}}
    \end{gather}
\end{subequations}
One may use the relations \eqref{eq:LinearSigmaModel_CouplingsMassesRelation} to rewrite the above equation in terms of the particle masses and the $\sigma$-vev $v_0$.
\begin{equation}
    \chi^2=\frac{-m_\sigma^2/2+\sqrt{3\epsilon m_\sigma^2/v_0^2+m_\sigma^4}}{3}=\frac{-m_\sigma^2+2m_\sigma\sqrt{3\epsilon/v_0^2+m_\sigma^2}}{6}\,,\qquad\rho=\sqrt{v_0\frac{2\chi^2+m_\sigma^2}{m_\sigma^2}}
\end{equation}
\todo{This is a number of order $m_{\Sigma}^2$. Of what order is $u_\mu$ (in $\text{eV}^{-1}$? But then again, isn't $u^\mu$ normalized to $\pm 1$?)}

In Milne coordinates $(x^\mu)=(\tau,r,\varphi,\eta_s)$ the $4$-velocity has components $(u^\mu)=(\gamma,\gamma v,0,0)$, where $v=v(\alpha)$ is a function on the freeze-out surface. \todo{Are these upstairs or downstairs indexed coordinates?}

\subsection{Conserved Currents from Chiral Symmetry}

Define
\begin{equation}
    \Phi=\sigma\mathbb{1}+\imagu\pi^a\tau^a=\begin{pmatrix}
        \sigma+\imagu\pi^0 & \sqrt{2}\pi^+      \\
        \sqrt{2}\pi^-      & \sigma-\imagu\pi^0
    \end{pmatrix}\,,\qquad\pi^\pm=\frac{1}{\sqrt{2}}(\pi^1\mp\imagu\pi^2)
\end{equation}
where $\tau^a$, $a\in\{0,1,2\}$ are the Pauli matrices (more precisely, $\tau^a=\sigma^{a+1}$ since $\sigma^0$ is usually reserved for the identity). Using $(\tau^a)^\dagger=\tau^a$, $\Tr(\tau^a\tau^b)=2\delta^{ab}$ and $\Tr(\tau^a)=0$ one immediately finds for example $\Tr(\Phi^\dagger\Phi)\equiv[\Phi^\dagger]_{jk}[\Phi]_{kj}=2(\sigma^2+\pi^a\pi^a)$ and the formerly stated linear $\sigma$-model Lagrangian can easily be shown to be equivalent to
\begin{equation}
    \mathscr{L}=-\frac{1}{2}\Tr\big[(\partial_\mu\Phi^\dagger)(\partial_\mu\Phi\big)]-\frac{1}{2}m^2\Tr\big[\Phi^\dagger\Phi\big]-\frac{\lambda}{4}\Big(\Tr\big[\Phi^\dagger\Phi\big]\Big)^2-\frac{\epsilon}{2}\Tr\big[\Phi^\dagger+\Phi\big]
    \label{eq:Lagrangian_LinearSigma2}
\end{equation}

Investigate the transformation behaviour of the field $\Phi$ under chiral symmetry transformation, following \cite{Koch_1997}. These symmetries are
\begin{subequations}
    \begin{align}
         &  & U_V: &  & \psi             & \mapsto e^{-\frac{\imagu}{2}\alpha_a\tau_a}\psi                                             &        &                                  \\
         &  & U_A: &  & \psi             & \mapsto e^{-\frac{\imagu}{2}\gamma_5\alpha_a\tau_a}\psi                                     &        &                                  \\
        \intertext{Pions and the $\sigma$ field are certain bound states of quarks, namely $\pi_a=\imagu\overline{\psi}\tau_a\gamma_5\psi$ and $\sigma=\overline{\psi}\psi$. Under the above transformations one finds infinitesimally}
         &  & U_V: &  & \pi_a            & \mapsto\pi_a+\epsilon_{abc}\alpha_b\pi_c                                                    & \sigma & \mapsto\sigma               &  & \\
         &  & U_A: &  & \pi_a            & \mapsto\pi_a+\alpha_a\sigma                                                                 & \sigma & \mapsto\sigma-\pi_a\alpha_a &  & \\
        \intertext{and after some calculations}
         &  & U_V: &  & \Phi^{(\dagger)} & \mapsto \Phi^{(\dagger)}-\imagu\frac{\alpha^a}{2}[\tau^a,\Phi^{(\dagger)}]                                                              \\
         &  & U_A: &  & \Phi^{(\dagger)} & \mapsto \Phi^{(\dagger)}\overset{(-)}{+}\imagu\frac{\alpha^a}{2}\{\tau^a,\Phi^{(\dagger)}\} &        &                                  \\
        \intertext{This infintesimal transformation behaviour corresponds to the finite transformations}
         &  & U_V: &  & \Phi^{(\dagger)} & \mapsto U\Phi^{(\dagger)}U^\dagger                                                          &        &                                  \\
         &  & U_A: &  & \Phi             & \mapsto U^\dagger\Phi U^\dagger\,,\quad \Phi^\dagger\mapsto U\Phi^\dagger U                 &        &
    \end{align}
\end{subequations}
with $U=\exp\big(-\frac{\imagu}{2}\alpha^a\tau^a)$. Finally, the conserved currents are
\begin{subequations}
    \begin{align}
        J^\mu_V & =-\frac{\partial\mathscr{L}}{\partial[\partial_\mu\Phi]_{jk}}[\delta\Phi]_{jk}-\frac{\partial\mathscr{L}}{\partial[\partial_\mu\Phi^\dagger]_{jk}}[\delta\Phi^\dagger]_{jk}                                                                      \\
                & =\frac{-\imagu\alpha^a}{2}\frac{1}{2}\Big([\partial^\mu\Phi^\dagger]_{kj}[\tau^a,\Phi]_{jk}+[\partial^\mu\Phi]_{kj}[\tau^a,\Phi^\dagger]_{jk}\Big)                                                                                               \\
                & =\frac{-\imagu\alpha^a}{2}\frac{1}{2}\Tr\Big((\partial_\mu\Phi^\dagger)[\tau^a,\Phi]+(\partial_\mu\Phi)[\tau^a,\Phi^\dagger]\Big)                                                                                                                \\
                & =\frac{-\imagu\alpha^a}{2}\frac{1}{2}\Tr\Big(\big((\partial_\mu\sigma)-\imagu\tau^d(\partial_\mu\pi^d)\big)\cdot2\epsilon^{abc}\tau^b\pi^c+\big((\partial_\mu\sigma)+\imagu\tau^d(\partial_\mu\pi^d)\big)\cdot(-2)\epsilon^{abc}\tau^b\pi^c\Big) \\
                & =\frac{-\imagu\alpha^a}{2}\frac{1}{2}\cdot\big(-8\imagu\epsilon^{abc}(\partial_\mu\pi^b)\pi^c\big)
    \end{align}
    and
    \begin{align}
        J^\mu_A & =-\frac{\partial\mathscr{L}}{\partial[\partial_\mu\Phi]_{jk}}[\delta\Phi]_{jk}-\frac{\partial\mathscr{L}}{\partial[\partial_\mu\Phi^\dagger]_{jk}}[\delta\Phi^\dagger]_{jk}                                                                    \\
                & =\frac{\imagu\alpha^a}{2}\frac{1}{2}\Big([\partial^\mu\Phi^\dagger]_{kj}\{\tau^a,\Phi\}_{jk}-[\partial^\mu\Phi]_{kj}\{\tau^a,\Phi^\dagger\}_{jk}\Big)                                                                                          \\
                & =\frac{\imagu\alpha^a}{2}\frac{1}{2}\Tr\Big((\partial_\mu\Phi^\dagger)\{\tau^a,\Phi\}-(\partial_\mu\Phi)\{\tau^a,\Phi^\dagger\}\Big)                                                                                                           \\
                & =\frac{\imagu\alpha^a}{2}\frac{1}{2}\Tr\Big(\big((\partial_\mu\sigma)-\imagu\tau^d(\partial_\mu\pi^d)\big)\cdot2(\sigma\tau^a+\imagu\pi^a)-\big((\partial_\mu\sigma)+\imagu\tau^d(\partial_\mu\pi^d)\big)\cdot2(\sigma\tau^a-\imagu\pi^a)\Big) \\
                & =\frac{\imagu\alpha^a}{2}\frac{1}{2}\cdot\big(4\imagu(\partial_\mu\sigma)\pi^a\Tr(\mathbb{1})-4\imagu\sigma(\partial_\mu\pi^b)\Tr(\tau^a\tau^b)\big)                                                                                           \\
                & =\frac{\imagu\alpha^a}{2}\frac{1}{2}\cdot8\imagu\cdot\big((\partial_\mu\sigma)\pi^a-\sigma(\partial_\mu\pi^a)\big)
    \end{align}
\end{subequations}

By equivalence of the Lagrangians \eqref{eq:Lagrangian_LinearSigma1} and \eqref{eq:Lagrangian_LinearSigma2} the equations of motion for $(\sigma,\pi^a)$, $a\in\{0,1,2\}$ read
\begin{subequations}
    \begin{align}
        -\partial_\mu\partial^\mu\sigma & =m^2\sigma-\frac{\lambda}{2}(\sigma^2+\pi^a\pi^a)\sigma-\epsilon \\
        -\partial_\mu\partial^\mu\pi^a  & =m^2\pi^a-\frac{\lambda}{2}(\sigma^2+\pi^b\pi^b)\pi^a
    \end{align}
\end{subequations}


\section{Finding the Spectrum at the Detector Surface}

In general the particle number $N$ and particle number density $n(\vec{x})$ in position space and $n(\vec{p})$ in momentum space associated to the condensate $\phi(\vec{x})$ of a complex scalar field are given by the relations
\begin{subequations}
    \begin{gather}
        n(\vec{x})=\phi(\vec{x})\phi^*(\vec{x})\,,\qquad n(\vec{p})=\phi(\vec{p})\phi^*(\vec{p})\\
        N=\int\dt^3xn(\vec{x})=\int\frac{\dt^3p}{(2\pi)^3}n(\vec{p})
    \end{gather}
\end{subequations}
with the convention $\phi(\vec{x})=\int\dt^3p/(2\pi)^3\phi(\vec{p})e^{-\imagu\vec{p}\vec{x}}$ for the Fourier transform.


% The field configuration found by this translation prescription is then propagated forwards in time by means of the retarded Greens function
% \begin{equation}
%     \overline{\phi}(x)=\langle\phi_a(x)\rangle=\int\dt^dyG_{\text{ret}}(x,y)j(y)
% \end{equation}
% $j(y)$ is defined to be the field configruation on the freezout surface. The retarded Greens function is given by $G_{\text{ret}}(x,y)=\Theta(x^0-y^0)\big(D(x-y)-D(y-x)\big)$ where
% \begin{equation}
%     D(\Delta x)=\int\frac{\dt^3p}{(2\pi)^3}\frac{1}{2\omega_\vec{p}}e^{-\imagu(\omega_\vec{p}\Delta t-\vec{p}\Delta\vec{x})}
% \end{equation}
% and $\omega_\vec{p}=\sqrt{m^2+\vec{p}^2}$.

% \includepdf{resources/notes_0430.pdf}

\subsection{Treating the Freeze Out Field as Source Term???}

Let's evaluate the Fourier transform of the condensate field. The calculation uses the result from linear response theory for the deviation of the expectation value $\overline{\phi}$ induced by a source term $j$, which we for the moment we assume to be specified by the hydro variables on the freezeout surface.
\begin{subequations}
    \begin{equation}
        \overline{\phi}(x)\equiv\langle\phi(x)\rangle=\int\dt^4G_{\text{ret}}(x,y)j(y)
    \end{equation}
    The retarded Greens function is given by
    \begin{equation}
        G_{\text{ret}}(x,y)=\Theta(x^0-y^0)\big[D(x-y)-D(y-x)\big]
    \end{equation}
    with the propagator
    \begin{equation}
        D(x-y)=\int\frac{\dt^3q}{(2\pi)^3}\frac{e^{-\imagu[\omega_{\vec{q}}(x^0-y^0)-\vec{q}(\vec{x}-\vec{y})]}}{2\omega_{\vec{q}}}
    \end{equation}
    and the relations
    \begin{equation}
        \int\dt^3x e^{\imagu\vec{x}(\vec{p}-\vec{q})}=(2\pi)^3\delta^{(3)}(\vec{p}-\vec{q})\,,\qquad\phi(t,\vec{p})=\int\dt^3x e^{\imagu\vec{p}\vec{x}}\phi(t,\vec{x})
    \end{equation}
\end{subequations}
Putting things together
\begin{subequations}
    \begin{align}
        \overline{\phi}(x^0\equiv t,\vec{p}) & =\int\dt^3x e^{\imagu\vec{p}\vec{x}}\overline{\phi}(x^0\equiv t,\vec{x})                                                                                                                                                                                                                                                           \\
                                             & =\int\dt^3xe^{\imagu\vec{p}\vec{x}}\int\dt^4yG_{\text{ret}}(x,y)j(y)                                                                                                                                                                                                                                                               \\
                                             & =\int\dt^3xe^{\imagu\vec{p}\vec{x}}\int\dt^4y\Theta(x^0-y^0)\big[D(x-y)-D(y-x)\big]j(y)                                                                                                                                                                                                                                            \\
                                             & =\int\dt^3xe^{\imagu\vec{p}\vec{x}}\int\dt^4y\int\frac{\dt^3q}{(2\pi)^3}\frac{1}{2\omega_{\vec{q}}}\Theta(x^0-y^0)\times\nonumber                                                                                                                                                                                                  \\
                                             & \phantom{=}\qquad\times\big[e^{-\imagu[w_{\vec{q}}(x^0-y^0)-\vec{q}(\vec{x}-\vec{y})]}-e^{\imagu[w_{\vec{q}}(x^0-y^0)-\vec{q}(\vec{x}-\vec{y})]}\big]j(y)                                                                                                                                                                          \\
                                             & =\int\dt^3x\int\dt^4y\int\frac{\dt^3q}{(2\pi)^3}\frac{1}{2\omega_{\vec{q}}}\Theta(x^0-y^0)\times\nonumber                                                                                                                                                                                                                          \\
                                             & \phantom{=}\qquad\times\big[e^{-\imagu[w_{\vec{q}}(x^0-y^0)+\vec{q}\vec{y}]}e^{\imagu(\vec{p}+\vec{q})\vec{x}}-e^{\imagu[w_{\vec{q}}(x^0-y^0)+\vec{q}\vec{y}]}e^{\imagu(\vec{p}-\vec{q})\vec{x}}\big]j(y)                                                                                                                          \\
                                             & =\int\dt^4y\int\dt^3q\frac{1}{2\omega_{\vec{q}}}\Theta(x^0-y^0)\times\nonumber                                                                                                                                                                                                                                                     \\
                                             & \phantom{=}\qquad\times\big[e^{-\imagu[w_{\vec{q}}(x^0-y^0)+\vec{q}\vec{y}]}\delta^{(3)}(\vec{p}+\vec{q})-e^{\imagu[w_{\vec{q}}(x^0-y^0)+\vec{q}\vec{y}]}\delta^{(3)}(\vec{p}-\vec{q})\big]j(y)                                                                                                                                    \\
                                             & =\int\dt^4y\frac{1}{2\omega_{\vec{p}}}\Theta(x^0-y^0)\big[e^{-\imagu[w_{\vec{p}}(x^0-y^0)-\vec{p}\vec{y}]}-e^{\imagu[w_{\vec{p}}(x^0-y^0)+\vec{p}\vec{y}]}\big]j(y)                                                                                                                                                                \\
                                             & =\int\dt^4y\frac{1}{2\omega_{\vec{p}}}\Theta(x^0-y^0)e^{\imagu\vec{p}\vec{y}}\big[e^{-\imagu\omega_{\vec{p}}(x^0-y^0)}-e^{\imagu\omega_{\vec{p}}(x^0-y^0)}\big]j(y)                                                                                                                                                                \\
                                             & =\int\dt^4y\frac{1}{\imagu\omega_{\vec{p}}}\Theta(x^0-y^0)e^{\imagu\vec{p}\vec{y}}\sin(\omega_{\vec{p}}(x^0-y^0))j(y)                                                                                                                                                                                                              \\
        \intertext{Specify this for the relevant case $\text{supp}j=\Sigma_{\text{freeze-out}}$ and parametrize the integral over the hypersurface $\Sigma$ via (see \eqref{calc:HypersurfaceMetric})$\int_\Sigma\dt^3\Sigma=\int\dt\varphi\dt\eta\dt\alpha r(\alpha)\tau(\alpha)\sqrt{r^{\prime 2}(\alpha)-\tau^{\prime 2}(\alpha)}$}
        \overline{\phi}(x^0\equiv t,\vec{p}) & =\int_0^{2\pi}\dt\varphi\int_{-\infty}^\infty\dt\eta\int_0^\pi\dt\alpha r(\alpha)\tau(\alpha)\sqrt{r^{\prime 2}(\alpha)-\tau^{\prime 2}(\alpha)}\frac{1}{\imagu\omega_{\vec{p}}}\Theta(x^0-\tau(\alpha)\cosh\eta)\times                                                                                                  \nonumber \\
                                             & \phantom{=}\qquad\times\exp\left(\imagu[\vec{p}_\perp\vec{x}_\perp+p_z\tau(\alpha)\sinh\eta]\right)\sin(\omega_{\vec{p}}(x^0-\tau(\alpha)\cosh\eta))j(\tau(\alpha),r(\alpha))                                                                                                                                                      \\
        \intertext{Performing the $\varphi$-integration, with $\varphi$ appearing in the integrand in $\vec{x}_\perp=(r\cos\varphi,r\sin\varphi)$, we choose to align $\varphi=0$ with the $p_x$ direction. Then $\vec{p}_\perp\vec{x}_\perp=p^\perp r\cos\varphi$. Use the Bessel function $J_0$ of first kind to simplify}
        \overline{\phi}(x^0\equiv t,\vec{p}) & =\int_{-\infty}^\infty\dt\eta\int_0^\pi\dt\alpha r(\alpha)\tau(\alpha)\sqrt{r^{\prime 2}(\alpha)-\tau^{\prime 2}(\alpha)}\frac{1}{\imagu\omega_{\vec{p}}}\Theta(x^0-\tau(\alpha)\cosh\eta)\times                                                                                                  \nonumber                        \\
                                             & \phantom{=}\qquad\times 2\pi J_0(p^\perp r(\alpha))\exp\left(\imagu p_z\tau(\alpha)\sinh\eta\right)\sin(\omega_{\vec{p}}(x^0-\tau(\alpha)\cosh\eta))j(\tau(\alpha),r(\alpha))
    \end{align}
\end{subequations}

\begin{calc}[Metric on Hypersurface]{calc:HypersurfaceMetric}
    Recall the metric $g_{\mu\nu}=\text{diag}(-1,1,\tau^2,r^2)$ in coordinates $(\tau,r,\eta,\varphi)$. Orthonormal tangent vectors to the freeze out hypersurface are $(\hat\partial_\varphi)^\mu=(0,0,0,r^{-1})=r^{-1}(\partial_\varphi)^\mu$, $(\hat\partial_\eta)^\mu=(0,0,\tau^{-1},0)=\tau^{-1}(\partial_\eta)^\mu$ and $(\hat\partial_\alpha)^\mu=\sqrt{r^{\prime 2}(\alpha)-\tau^{\prime 2}(\alpha)}^{-1}(\tau^{\prime}(\alpha),r^{\prime}(\alpha),0,0)=D(\alpha)(\partial_\alpha)^\mu$ with $D(\alpha)=\sqrt{r^{\prime 2}(\alpha)-\tau^{\prime 2}(\alpha)}^{-1}$. The projector on the hypersurface is
    \begin{equation}
        \gamma_{\mu\nu}=(\hat\partial_\varphi)_\mu(\hat\partial_\varphi)_\nu+(\hat\partial_\eta)_\mu(\hat\partial_\eta)_\nu+(\hat\partial_\alpha)_\mu(\hat\partial_\alpha)_\nu=\begin{pmatrix}
            D^2(\alpha)\tau^{\prime2}(\alpha)               & -D^2(\alpha)\tau^\prime(\alpha)r^\prime(\alpha) & 0      & 0   \\
            -D^2(\alpha)\tau^\prime(\alpha)r^\prime(\alpha) & D^2(\alpha)r^{\prime2}(\alpha)                  & 0      & 0   \\
            0                                               & 0                                               & \tau^2 & 0   \\
            0                                               & 0                                               & 0      & r^2
        \end{pmatrix}
    \end{equation}
    The normal of the hypersurface is $n^\mu\equiv(\hat\partial_\alpha^\perp)^\mu=D(\alpha)(r^\prime(\alpha),\tau^\prime(\alpha),0,0)$ and is timelike where $D$ is real. Naturally $\gamma_{\mu\nu}n^\nu=0$. In the basis $(\partial_\alpha,\partial_\eta,\partial_\varphi,n)$ using (in short form)
    \begin{equation}
        (\partial_\alpha)^\nu\gamma_{\mu\nu}(\partial_\alpha)^\mu=\begin{pmatrix}
            \tau^\prime \\r^\prime
        \end{pmatrix}^T\begin{pmatrix}
            -\tau^\prime \\
            r^\prime
        \end{pmatrix}=D^{-2}
    \end{equation}
    the hypersurface metric in coordinates $x^i=(\alpha,\eta,\varphi)$ reads
    \begin{equation}
        \gamma_{ij}=\text{diag}(D^{-2}(\alpha),\tau^2(\alpha),r^2(\alpha))
    \end{equation}
    and the volume element is given by $\dt\Sigma=r(\alpha)\tau(\alpha) D^{-1}(\alpha)\dt\alpha\dt\eta\dt\varphi$. The oriented surface element is \begin{equation}
        \dt\Sigma^\mu=n^\mu\dt\Sigma=r(\alpha)\tau(\alpha)(r^\prime(\alpha),\tau^\prime(\alpha),0,0)\dt\alpha\dt\eta\dt\varphi
    \end{equation}
\end{calc}

\subsection{Converting Spectra between Coordinate Systems}

Consider the coordinate change in momentum space
\begin{equation}
    \left\{
    \begin{split}
        p_x&=p^\perp\cos\varphi_p\\
        p_y&=p^\perp\sin\varphi_p\\
        p_z&=m^\perp\sinh\eta_p\\
        p_t&=m^\perp\cosh\eta_p
    \end{split}
    \right.\qquad\iff\qquad
    \left\{
    \begin{split}
        p^\perp&=\sqrt{p_x^2+p_y^2}\\
        \varphi_p&=\arctan(p_y/p_y)\\
        m^\perp&=\sqrt{p_t^2-p_z^2}\\
        \eta_p&=\artanh(p_z/p_t)
    \end{split}
    \right.
\end{equation}
with Jacobian
\begin{equation}
    \big\vert\frac{\partial(p^\perp,\varphi_p,m^\perp,\eta_p)}{\partial(p_x,p_y,p_z,p_t)}\big\vert=\frac{1}{m^\perp p^\perp}
\end{equation}
Let $f(p_\mu)$ be some distribution function and $F$ its momentum space integral evaluated on the momentum shell and future directed momenta.

\begin{subequations}
    \begin{align}
        F=\int\frac{\dt^4p_{\text{cart}}}{(2\pi)^4}\delta(p^2-m^2)\Theta(p_t)f(p_\mu) & =\int\frac{\dt p_t}{2\pi}\int\frac{\dt^3p_{\text{cart}}}{(2\pi)^3}\frac{1}{2\omega_{\vec{p}}}\big(\delta(p_t-\omega_{\vec{p}})+\delta(p_t+\omega_{\vec{p}})\big)\Theta(p_t)f(p_\mu) \\
                                                                                      & =\frac{1}{2\pi}\int\frac{\dt^3p_{\text{cart}}}{(2\pi)^3}\frac{1}{2\omega_{\vec{p}}}f(p_\mu)\big\vert_{p_t=\omega_{\vec{p}}}
    \end{align}
    On the other hand
    \begin{align}
        F=\int\frac{\dt^4p_{\text{cart}}}{(2\pi)^4}\delta(p^2-m^2)\Theta(p_t)f(p_\mu) & =\frac{1}{(2\pi)^4}\int_0^\infty \dt p^\perp\int_0^\infty \dt m^\perp\int_{-\infty}^\infty \dt\eta_p\int_0^{2\pi}\dt\varphi_pm^\perp p^\perp\times\nonumber \\
                                                                                      & \phantom{=}\qquad\times\delta((p^\perp)^2-(m^\perp)^2)f(p_\mu)                                                                                              \\
        \intertext{assume $f(p^\mu)=f(p^\perp,m^\perp,\eta_p)$ and perform the $\varphi$-integraion}
                                                                                      & =\frac{1}{(2\pi)^3}\int_0^\infty\dt m^\perp\int_{-\infty}^\infty\dt\eta_p\frac{m^\perp}{2}f(p_\mu)\big\vert_{m^\perp=p^\perp}
    \end{align}
    leding to the important result
    \begin{equation}
        \frac{1}{2\pi}\int\frac{\dt^3p_{\text{cart}}}{(2\pi)^3}\frac{1}{2\omega_{\vec{p}}}f(p_\mu)\big\vert_{p_t=\omega_{\vec{p}}}=\frac{1}{(2\pi)^3}\int_0^\infty\dt m^\perp\int_{-\infty}^\infty\dt\eta_p\frac{m^\perp}{2}f(p_\mu)\big\vert_{m^\perp=p^\perp}
    \end{equation}
\end{subequations}

Since the restrictions $p_t=\omega_{\vec{p}}$ and $m^\perp=p^\perp$ are equivalent (considering the parametrization that already satisfies $p_t=p^\perp\cosh\eta_p\geq 0$) we find
\begin{equation}
    \omega_{\vec{p}}\frac{\dt F}{\dt p_x\dt p_y\dt p_z}=\frac{1}{2\pi m^\perp}\frac{\dt F}{\dt m^\perp\dt\eta_p}
\end{equation}
The result applies to the case
\begin{equation}
    f(p_\mu)\big\vert_{p_t=\omega_{\vec{p}}}=2\omega_{\vec{p}}\cdot 2\pi\cdot n(\vec{p})
\end{equation}
and $F=N$.


\subsection{Fourier Decomposition on Freeze Out Surface???}

Generally a Fourier Ddecomposition to solve the Klein-Gordon equation in terms of $3$-momentum-modes is given by
\begin{equation}
    \phi(x^\mu)=\int\frac{\dt^4p}{(2\pi)^4}\tilde{\phi}(\vec{p})e^{\imagu p_\mu x^\mu}\delta(p^2-m^2)\Theta(p_0)+\text{c.c.}=\frac{1}{2\pi}\int\frac{\dt^3p}{(2\pi)^3}\frac{1}{2\omega_{\vec{p}}}\phi(\vec{p})e^{\imagu(\omega_{\vec{p}}x^0-\vec{p}\vec{x})}+\text{c.c.}
\end{equation}
For simplicity neglect the $+\text{c.c.}$.

Considering a hypersurface $\Sigma$ and using the field data given only on this hypersurface - that is considering the restriction $\phi\vert_{x\in\Sigma}$ - can we reconstruct $\tilde{\phi}(\vec{p})$? The map $\tilde{\phi}(\vec{p})\mapsto\phi(x^\mu\in\Sigma)$ is trivially given by the mode decomposition above. Let $\Sigma=\Sigma_t=\{x^\mu\in\mathbb{R}^{(1,3)}\vert x^0=t=\text{const}\}$ be a slice of constant lab time. Then the map $\phi(x^\mu\in\Sigma_t)\equiv \phi_t(\vec{x})\mapsto\tilde{\phi}(\vec{p})$ is easily found to be
\begin{equation}
    \phi(\vec{p})=(2\pi)(2\omega_{\vec{p}})\int_{\Sigma_t}\dt^3x\phi_t(\vec{x})e^{-\imagu(\omega_{\vec{p}}t-\vec{p}\vec{x})}
\end{equation}
which of course uses the orthogonality relation
\begin{equation}
    \int\dt^nx_{\text{cart}}e^{\imagu (p_\mu-q_\mu)x^\mu}=(2\pi)^n\delta^{(n)}(p-q)
\end{equation}
valid in cartesian coordinates.

We would like to generalize this to arbitrary $\Sigma$. Let $(y^i)_{i=1}^3$ be the coordinates of a parametrization $x^\mu(y^i)$ of $\Sigma$. The naive attempt would be an integral of the form
\begin{equation}
    \tilde{\phi}(\vec{p})\overset{?}{=}(2\pi)(2\omega_{\vec{p}})\int_\Sigma\dt^3y\sqrt{\gamma}\phi(x^\mu(y^i))e^{-\imagu(\omega_{\vec{p}}t(y^i)-\vec{p}\vec{x}(y^i))}
\end{equation}
where $\gamma$ is the induced metric determinant. The relevant example is $\Sigma=\{x^\mu\in\mathbb{R}^{(1,3)}\vert (\tau,r)=(\tau(\alpha),r(\alpha))\}$ with $\tau,r$ defined by the coordinate transformation
\begin{equation}
    \left\{\begin{split}
        t&=\tau\cosh\eta\\
        z&=\tau\sinh\eta\\
        x&=r\cos\varphi\\
        y&=r\sin\varphi
    \end{split}\right.
    \qquad\iff\qquad
    \left\{\begin{split}
        \tau&=\sqrt{t^2-z^2}\\
        \eta&=\artanh(z/t)\\
        r&=\sqrt{x^2+y^2}\\
        \varphi&=\arctan(y/x)
    \end{split}\right.
\end{equation}

\subsubsection{Fourier Transform in Adapted Coordinates}

Let's first evaluate $p_\mu x^\mu$ in the Bjorken coordinate system. Therefore introduce an analogous coordinate change in momentum space
\begin{equation}
    \left\{\begin{split}
        p_t&=m_\perp\cosh\eta_p\\
        p_z&=m_\perp\sinh\eta_p\\
        p_x&=p_\perp\cos\varphi_p\\
        p_y&=p_\perp\sin\varphi_p
    \end{split}\right.
\end{equation}
to rewrite the scalar product as
\begin{equation}
    p_\mu x^\mu\equiv\tau(p_t\cosh\eta-p_z\sinh\eta)-r(p_x\cos\varphi+p_y\sin\varphi)=\tau m_\perp\cosh(\eta-\eta_p)-r p_\perp\cos(\varphi-\varphi_p)
\end{equation}
We used the identities
\begin{equation}
    \cosh(a-b)=\cosh a\cosh b-\sinh a\sinh b\,,\qquad\cos(a-b)=\cos a\cos b+\sin a\sin b
\end{equation}
The integral measure changes according to $\dt^4p_{\text{cart}}=\dt m_\perp\dt p_\perp\dt\eta_p\dt\varphi_p\cdot m_\perp p_\perp$. The momentum shell condition $p^2=m^2$ is equivalently parametrized by $m_\perp^2=p_\perp^2+m^2\eqdef \omega_\perp^2$.

These coordinates are adapted to boost symmetry $\eta\to\eta^\prime$ along the beam direction and rotational symmetry $\varphi\to\varphi^\prime$ around the beam axis. Investigate first the implications on the mode decomposition, by requesting that $(\partial/\partial\eta)\phi(x^\mu)=0=(\partial/\partial\varphi)\phi(x^\mu)$.
\begin{subequations}
    \begin{align}
        \phi(x^\mu) & =\frac{1}{(2\pi)^4}\int_{-\infty}^\infty\dt m_\perp\int_0^\infty\dt p_\perp\int_{-\infty}^\infty\dt\eta_p\int_0^{2\pi}\dt\varphi_p\cdot m_\perp p_\perp\delta(m_\perp^2-\omega_\perp^2)\Theta(m_\perp)\times\nonumber \\
                    & \phantom{=}\qquad\times \tilde{\phi}(p_x(p_\perp,\varphi_p),p_y(p_\perp,\varphi_p),p_z(m_\perp,\eta_p))e^{\imagu(\tau m_\perp\cosh(\eta-\eta_p)-r p_\perp\cos(\varphi-\varphi_p))}                                    \\
        \intertext{\dots shift $\varphi_p\to\varphi_p+\varphi$ and $\eta_p\to\eta_p+\eta$\dots}
                    & =\frac{1}{(2\pi)^4}\int_{-\infty}^\infty\dt m_\perp\int_0^\infty\dt p_\perp\int_{-\infty}^\infty\dt\eta_p\int_0^{2\pi}\dt\varphi_p\cdot m_\perp p_\perp\delta(m_\perp^2-\omega_\perp^2)\Theta(m_\perp)\times\nonumber \\
                    & \phantom{=}\qquad\times \tilde{\phi}(p_x(p_\perp,\varphi_p+\varphi),p_y(p_\perp,\varphi_p+\varphi),p_z(m_\perp,\eta_p+\eta))e^{\imagu(\tau m_\perp\cosh\eta_p-r p_\perp\cos\varphi_p)}                                \\
        \intertext{From this it follows that $\tilde{\phi}(p_x,p_y,p_z)=\tilde{\phi}(p_\perp)$ and we can simplify the integral. Let's also evaluate the $\delta$-distribution by using that $\delta(m_\perp^2-\omega_\perp^2)=(1/2\omega_\perp)(\delta(m_\perp-\omega_\perp)+\delta(m_\perp+\omega_\perp))$}
        \phi(x^\mu) & =\frac{1}{(2\pi)^4}\frac{1}{2}\int_0^\infty\dt p_\perp\int_{-\infty}^\infty\dt\eta_p\int_0^{2\pi}\dt\varphi_p\cdot p_\perp\tilde{\phi}(p_\perp)e^{\imagu(\tau \omega_\perp\cosh\eta_p-r p_\perp\cos\varphi_p)}        \\
                    & =\frac{1}{2}\frac{1}{(2\pi)^4}\int_0^\infty\dt p_\perp p_\perp\tilde{\phi}(p_\perp)\big(2\pi J_0(r p_\perp)\big)\big(\pi(-Y_0(\tau\omega_\perp)+\imagu J_0(\tau\omega_\perp))\big)
    \end{align}
\end{subequations}
where in the last step the following integral representation of Bessel functions of the first kind $J_0(x)$ and of the second kind $Y_0(x)$ where used \url{https://dlmf.nist.gov/10.9}
\begin{subequations}
    \begin{gather}
        J_0(x\in\mathbb{R})=\frac{1}{2\pi}\int_0^{2\pi}\dt t\exp{(\textcolor{red}{\overset{?}{\pm}}\imagu x\cos t)}\,,\quad
        J_0(x>0)=\frac{1}{\pi}\int_{-\infty}^\infty\dt t\sin(x\cosh t)\,,\quad Y_0(x>0)=-\frac{2}{\pi}\int_0^\infty\dt t\cos(x\cosh t)\\
        H_\nu^{(1)}(z)
        \int_0^{2\pi}\dt\varphi\int_{-\infty}^\infty\dt\eta e^{\imagu(a\cosh\eta-b\cos\varphi)}=\Big[2\pi J_0(b)\times\big(\pi(-Y_0(a)+\imagu J_0(a))\big)\Big]\\
        \intertext{Other relevant properties are}
        \frac{\dt}{\dt x}J_0(x)=-J_1(x)\,,\qquad\frac{\dt}{\dt x}Y_0(x)=-Y_1(x)\\
        H_\nu^{(1)}(z)=J_\nu(z)+\imagu Y_\nu(z)
        \label{eq:BesselFunctions}
    \end{gather}
\end{subequations}


% Let us finally come back to the naive idea of extracting a particular Fourier component by convolution of the field in position space with some plane wave. We wish to evaluate integrals of the form
% \begin{subequations}
%     \begin{align}
%         \phi(p_\mu) & \sim\int_\Sigma\dt^3y\sqrt{\gamma}\phi(x^\mu(y^i))\exp\big(\imagu[\tau(\alpha) m_\perp^p\cosh(\eta-\eta_p)-r(\alpha) p_\perp\cos(\varphi-\varphi_p)]\big)                                                                       \\
%                     & \sim\int_\Sigma\dt\alpha\dt\eta\dt\varphi\sqrt{\gamma}\Bigg(\frac{1}{{(2\pi)^4}}\int\dt m_\perp^q\dt q_\perp\dt \eta_q\dt \varphi_q\cdot m_\perp^q q_\perp\delta(m_\perp^2-\omega_\perp^2)\Theta(m_\perp)\phi(q_\perp)\times\nonumber \\
%                     & \phantom{=}\quad\times\exp\Big(\imagu\big[\tau(\alpha) (m_\perp^p\cosh(\eta-\eta_p)-m_\perp^q\cosh(\eta-\eta_q))-r(\alpha) (p_\perp\cos(\varphi-\varphi_p)-q_\perp\cos(\varphi-\varphi_q))\big]\Big)\Bigg)
%     \end{align}
% \end{subequations}
% Eventually $p_\mu$ should be some on-shell momentum. There seems to exist no useful orthonormality condition to extract $\phi(p_\perp)$ from this.

% In terms of Bessel function, let $(z_n)_{n\in\mathbb{N}}$ be zeros of $J_\nu$. Then \url{http://physics.ucsc.edu/~peter/116C/bess_orthog.pdf}
% \begin{equation}
%     \int_0^1\dt tJ_\nu(z_nt)J_\nu(z_mt)=\delta_{nm}\frac{1}{2}\frac{\dt J_\nu(x)}{\dt x}\big\vert_{x=z_n}
% \end{equation}

Naively, there seems to be no useful orthogonality relation\dots to extract $\tilde{\phi}(p_\perp)$ from this.

\subsubsection{Invariance of Fourier Transform w.r.t. Deformations of the Hypersurface}
\label{subsec:FourierDeformHypersurface}

Let $\phi_1,\phi_2$ be fields of equal mass evolving according to the KG equation. Then the current
\begin{equation}
    J_\mu[\phi_1,\phi_2]=-\imagu(\phi_1\partial_\mu\phi_2^*-(\partial_\mu\phi_1)\phi_2^*)\eqdef-\imagu\phi_1\overset{\leftrightarrow}{\partial_\mu}\phi_2
\end{equation}
is conserved. Recall Gauß law
\begin{equation}
    \int_\Omega\dt \Omega\nabla_\mu J^\mu=\int_{\partial\Omega}\dt\sigma_\mu J^\mu
\end{equation}
with $\dt\sigma_\mu$ the outwards oriented surface normal of the spacetime volume $\Omega$. The bilinear form
\begin{equation}
    (\phi_1,\phi_2)_\Sigma=\int_\Sigma\dt\Sigma_\mu J^\mu[\phi_1,\phi_2]=-\imagu\int_\Sigma\dt\Sigma_\mu \phi_1\overset{\leftrightarrow}{\partial^\mu}\phi_2^*
\end{equation}
is therefore independent of the choice of (Cauchy) hypersurface $\Sigma$ (if $\partial\Sigma$ is changed, one must carefully check for further contributions in Gauß law). Choose a hypersurface $\Sigma_t$ where $t=\text{const}$. Consider a solution to the KG equation given by its Fourier decomposition
\begin{equation}
    \phi(t,\vec{x})=\int\frac{\dt^3p}{(2\pi)^3}\tilde{\phi}(\vec{p})u_{\vec{p}}(t,\vec{x})\,,\qquad u_{\vec{p}}(t,\vec{x})=\frac{1}{\sqrt{2\omega_{\vec{p}}}}e^{-\imagu(\omega_{\vec{p}}t-\vec{p}\vec{x})}
\end{equation}
The mode functions $u_{\vec{p}}$ are orthogonal w.r.t. to the bilinear form $(\cdot,\cdot)_{\Sigma_t}$ and normalized according to
\begin{equation}
    (u_{\vec{p}},u_{\vec{q}})_{\Sigma_t}=(2\pi)^3\delta^{(3)}(\vec{p}-\vec{q})
\end{equation}
and the Fourier transform $\tilde{\phi}(\vec{p})$ can be extracted via $$\tilde{\phi}(\vec{p})=(\phi,u_{\vec{p}})_{\Sigma_t}$$ and can thus be evaluated on any Cauchy surface.

\begin{rmrk}[Particle Density w.r.t. Convention of Fourier Decomposition]{rmrm:ParticleDensityFourier}
    Let $n(t,\vec{x})=\abs{\phi(t,\vec{x})}^2$ and $N=\int\dt^3xn(\vec{x})$. Then
    \begin{equation}
        N=\int\dt^3x\int\frac{\dt^3p}{(2\pi)^3}\frac{\dt^3q}{(2\pi)^3}\tilde{\phi}(\vec{p})\frac{1}{2\sqrt{\omega_{\vec{p}}\omega_{\vec{q}}}}\tilde{\phi}^*(\vec{q})e^{-\imagu((\omega_{\vec{p}}-\omega_{\vec{q}})t-(\vec{p}-\vec{q})\vec{x})}=\int\frac{\dt^3p}{(2\pi)^3}\frac{1}{2\omega_{\vec{p}}}\abs{\tilde{\phi}(\vec{p})}^2
    \end{equation}
    leading to $n(\vec{p})=\frac{1}{2\omega_{\vec{p}}}\abs{\tilde{\phi}(\vec{p})}^2$ which is slightly different convention than before.
\end{rmrk}

\debugbox{
    \begin{minipage}{\linewidth}
        \centering
        \includegraphics[width=0.4\linewidth]{images/FreezeOutSurface.pdf}
        \captionof{figure}{Freezeout surface in $\tau$-$r$-plane\cite{KirchnerEtAl_2023}.}
        \label{fig:FreezeOutSurface_rtau}
    \end{minipage}
}

Consider the freezeout on the hypersurface depicted in \ref{fig:FreezeOutSurface_rtau}. Assume that the condensate contribution as a function in phase space $f_{\text{cond}}(x^\mu,\vec{p})$ vanishes on $\Sigma_{\text{\rom{2}}}$ and $\Sigma_{\text{\rom{5}}}$, i.e. is contained within the union of all light cones starting on the freeze out surface $\Sigma_{FO}\equiv\Sigma_{\text{\rom{1}}}\cap\Sigma_\text{{\rom{3}}}$. \todo{By causality this seems reasonable, but from Fourier decomposition of a classical field this is not at all clear.}


Following the reasoning from \cite{KirchnerEtAl_2023}, we wish to apply Gauß law. Consider separately the contribution on the $\tau$-axis
\begin{equation*}
    \int_{\Sigma_{r=0}}\dt\Sigma_\mu J^\mu\qquad\text{or}\qquad\lim_{r\to 0}\int_{\Sigma_{r}}\dt\Sigma_\mu J^\mu
\end{equation*}
The surface vector on this hypersurface is $\dt\Sigma_\mu=r\tau\dt\tau\dt\eta\dt\varphi(0,1,0,0)$ and thus vanishes at $r=0$ (the hypersurface $\Sigma_{r=0}$ has zero $3$-volume). Since the derivative in the integrad introduces no divergencies, the contribution of $\Sigma_{r=0}$ to Gauß law is zero.
% \begin{subequations}
%     \begin{align}
%         \partial_\mu=(\partial_t,\partial_x,\partial_y,\partial_z)                                                                                                              \\
%         \partial_x+\partial_y=(\frac{\partial r}{\partial x}+\frac{\partial r}{\partial y})\frac{\partial}{\partial r}                                                          \\
%         \hat{e}_x\partial_x+\hat{e}_y\partial_y=\hat{e}_x\frac{\partial r}{\partial x}\partial_r+\hat{e}_y\frac{\partial r}{\partial y}\partial_r\partial_r=\hat{e}_r\partial_r \\
%         =\cos\varphi\hat{e}_x\partial_r+\sin\varphi\hat{e}_y                                                                                                                    \\
%         \hat{e}_r=\cos\varphi\hat{e}_x+\sin\varphi\hat{e}_y                                                                                                                     \\
%         \hat{e}_\varphi=-\sin\varphi\hat{e}_x+\cos\varphi\hat{e}_y                                                                                                              \\
%         \hat{e}_x=\cos\varphi\hat{e}_r-\sin\varphi\hat{e}_\varphi                                                                                                               \\
%         \hat{e}_y=\sin\varphi\hat{e}_r+\cos\varphi\hat{e}_varphi                                                                                                                \\
%     \end{align}
% \end{subequations}

We can therefore write
\begin{equation}
    \tilde{\phi}(\vec{p})=(\phi,u_{\vec{p}})_{\Sigma_t}=(\phi,u_{\vec{p}})_{\Sigma_{\tau\gg\tau_L}}=(\phi,u_{\vec{p}})_{\Sigma_{FO}}
\end{equation}
The second "$=$" assumes that $\phi=0$ for large spacetime rapidities $\eta\to\pm\infty$ and the $\tau=\const$ hypersurface can be deformed to a $t=\const$ hypersurface.

Using the freezeout surface parametrization stated in earlier paragraphs one computes \todo{some errors}
\begin{subequations}
    \begin{align}
        \tilde{\phi}(\vec{p}) & =-\imagu\int_{-\infty}^\infty\dt\eta\int_0^{2\pi}\dt\varphi\int_0^\pi\dt\alpha\tau(\alpha) r(\alpha)\frac{1}{\sqrt{2\omega_{\vec{p}}}}\Bigg[r^\prime(\alpha)\phi(\tau,r)\overset{\leftrightarrow}{\partial_\tau}e^{\imagu(\tau \omega_\perp\cosh(\eta-\eta_p)-r p_\perp\cos(\varphi-\varphi_p))}+\nonumber \\
                              & \phantom{=}\qquad + \tau^\prime(\alpha)\phi(\tau,r)\overset{\leftrightarrow}{\partial_r}e^{\imagu(\tau \omega_\perp\cosh(\eta-\eta_p)-r p_\perp\cos(\varphi-\varphi_p))}\Bigg]                                                                                                                             \\
                              & =-\imagu\int_{-\infty}^\infty\dt\eta\int_0^{2\pi}\dt\varphi\int_0^\pi\dt\alpha\tau(\alpha) r(\alpha)\frac{1}{\sqrt{2\omega_{\vec{p}}}}\Bigg[r^\prime(\alpha)\phi(\tau,r)\overset{\leftrightarrow}{\partial_\tau}e^{\imagu(\tau \omega_\perp\cosh\eta-r p_\perp\cos\varphi)}+\nonumber                      \\
                              & \phantom{=}\qquad + \tau^\prime(\alpha)\phi(\tau,r)\overset{\leftrightarrow}{\partial_r}e^{\imagu(\tau \omega_\perp\cosh\eta-r p_\perp\cos\varphi)}\Bigg]                                                                                                                                                  \\
                              & =-\imagu\int_0^\pi\dt\alpha\tau(\alpha) r(\alpha)\frac{1}{\sqrt{2\omega_{\vec{p}}}}\Bigg[r^\prime(\alpha)\phi(\tau,r)\overset{\leftrightarrow}{\partial_\tau}\Big[2\pi J_0(r p_\perp)\times\pi\big(-Y_0(\tau\omega_\perp)+\imagu J_0(\tau\omega_\perp)\big)\Big]+\nonumber                                 \\
                              & \phantom{=}\qquad + \tau^\prime(\alpha)\phi(\tau,r)\overset{\leftrightarrow}{\partial_r}\Big[2\pi J_0(r p_\perp)\times\pi\big(-Y_0(\tau\omega_\perp)+\imagu J_0(\tau\omega_\perp)\big)\Big]\Bigg]                                                                                                          \\
                              & =-\imagu\int_0^\pi\dt\alpha\tau(\alpha) r(\alpha)\frac{1}{\sqrt{2\omega_{\vec{p}}}}\Bigg[-(r^\prime\partial_\tau+\tau^\prime\partial_r)\phi(\tau,r)[2\pi J_0(r p_\perp)\times\pi\big(-Y_0(\tau\omega_\perp)+\imagu J_0(\tau\omega_\perp)\big)\Big]+\nonumber                                               \\
                              & \phantom{=}\qquad + \phi(\tau,r)\Big[\tau^\prime\times2\pi p_\perp J_1(r p_\perp)\times\pi\big(-Y_0(\tau\omega_\perp)+\imagu J_0(\tau\omega_\perp)\big)+\nonumber                                                                                                                                          \\
                              & \phantom{=}\qquad\phantom{-\phi(\tau,r)\Big[}+r^\prime\times2\pi J_0(r p_\perp)\times\pi\omega_\perp\big(Y_1(\tau\omega_\perp)-\imagu J_0(\tau\omega_\perp)\big)\Big]\Bigg]                                                                                                                                \\
                              & =-\frac{2\pi^2\imagu}{\sqrt{2\omega_{\vec{p}}}}\int_0^\pi\dt\alpha\tau(\alpha) r(\alpha)\Bigg[-(r^\prime\partial_\tau+\tau^\prime\partial_r)\phi(\tau,r)[J_0(r p_\perp)\times\big(-Y_0(\tau\omega_\perp)+\imagu J_0(\tau\omega_\perp)\big)\Big]+\nonumber                                                  \\
                              & \phantom{=}\qquad + \phi(\tau,r)\Big[\tau^\prime\times p_\perp J_1(r p_\perp)\times\big(-Y_0(\tau\omega_\perp)+\imagu J_0(\tau\omega_\perp)\big)+\nonumber                                                                                                                                                 \\
                              & \phantom{=}\qquad\phantom{-\phi(\tau,r)\Big[}+r^\prime\times J_0(r p_\perp)\times\omega_\perp\big(Y_1(\tau\omega_\perp)-\imagu J_0(\tau\omega_\perp)\big)\Big]\Bigg]
    \end{align}
\end{subequations}
where the Bessel function identities \eqref{eq:BesselFunctions} were used. One may choose to work at mid-rapidity $\eta_p=0$ where $\omega_{\vec{p}}=m_\perp\vert_{\text{on-shell}}=\omega_\perp$. Note that for large $p_\perp$ one finds
\begin{equation}
    \omega_\perp\sim p_\perp\,,\qquad J_\nu,Y_\nu\sim\frac{1}{\sqrt{p_\perp}}
\end{equation}
and therefore $\phi(\vec{p})\sim\frac{1}{\sqrt{p_\perp}}$.

How to treat the projection $r^\prime\partial_\tau+\tau^\prime\partial_r\equiv\partial_\perp\propto n^\mu\partial_\mu$ of $\partial_\mu$ onto the normal vector $\propto n^\mu$? One could argue that $\partial_\perp\phi$ not specified by field data on the hypersurface and one needs to find suitable initial conditions. A reasonable choice would be $\pi\vert_{\Sigma_{FO}}\equiv\dot{\phi}\vert_{\Sigma_{FO}}=0$. Since $\partial_\eta\phi=0$ by symmetry assumption, this implies $\partial_\tau\phi\vert_{\Sigma_{FO}}=0$. The derivatives $\partial_\alpha\equiv\tau^\prime\partial_\tau+r^\prime\partial_r$ and $\partial_\perp\equiv r^\prime\partial_\tau+\tau^\prime\partial_r$ therefore can be related by $\partial_\perp\phi\vert_{\Sigma_{FO}}=\frac{\tau^\prime}{r^\prime}\partial_\alpha\phi\vert_{\Sigma_{FO}}$.

\subsection{Properly Considering all Initial Conditions}

Adapt the reasoning in \cite{Amelino-CameliaEtAl_1997}. The idea is to model a non-linear, particle producing interaction process with the inhomogeneous KG equation. After all interactions have decayed and the source has vanished, the particle are effectively free. One can use the true e.o.m. to evolve initial conditions and then, after all interactions have decayed, reconstruct the hypothetical source term and derive particle production from this. For a field with the e.o.m.
\begin{equation}
    (\Box-m^2)\phi(x)=J(x)
\end{equation}
the particle spectrum after the source has vanished is
\begin{equation}
    2\omega_{\vec{p}}\frac{\dt N}{\dt^3\vec{p}}=\frac{1}{(2\pi)^3}\abs{\tilde{J}(\vec{p})}^2
\end{equation}

Use following conventions:
\begin{impt}[Conventions for Fourier Transform]{impt:FourierConvention_PionProdPaper}
    Signature is $(-,+,+,+)$.
    \begin{subequations}
        \begin{align}
            \tilde{f}(p)               & =\int\dt^4x e^{-\imagu px}f(x)                     \\
            \tilde{f}^{(3)}(t,\vec{p}) & =\int\dt^3xe^{-\imagu\vec{p}{\vec{x}}}f(t,\vec{x}) \\
            \tilde{J}(\vec{p})         & =\tilde{J}(p)\vert_{p^0=\omega_{\vec{p}}}
        \end{align}
    \end{subequations}
\end{impt}

The full solution of a KG-field is specified by 2 initial conditions, e.g. $\phi(t,\vec{x})$ and $\dot{\phi}(t,\vec{x})$ for fixed $t$, or equivalently 2 spectral functions $a(p)$, $b(p)$  within the following decomposition:
\begin{subequations}
    \begin{align}
        \phi(t,\vec{x})       & =2\pi\int\frac{\dt^4p}{(2\pi)^4}\delta(p^2-m^2)\Theta(p^0)\big(a(p)e^{\imagu px}+b(p)e^{-\imagu px}\big)                                                                                                          \\
                              & =\int\frac{\dt^3p}{(2\pi)^3}\frac{1}{2\omega_{\vec{p}}}\big(a(\omega_{\vec{p}},\vec{p})e^{-\imagu(\omega_{\vec{p}}t-\vec{p}\vec{x})}+b(\omega_{\vec{p}},\vec{p})e^{\imagu(\omega_{\vec{p}}t-\vec{p}\vec{x})}\big) \\
        \dot{\phi}(t,\vec{x}) & =\int\frac{\dt^3p}{(2\pi)^3}\frac{1}{2}\big(-\imagu a(\omega_{\vec{p}},\vec{p})e^{-\imagu(\omega_{\vec{p}}t-\vec{p}\vec{x})}+\imagu  b(\omega_{\vec{p}},\vec{p})e^{\imagu(\omega_{\vec{p}}t-\vec{p}\vec{x})}\big)
    \end{align}
    From computing the transformations
    \begin{align}
        \tilde{\phi}^{(3)}(t,\vec{p})=\int\dt^3x e^{-\imagu\vec{p}\vec{x}}\phi(t,\vec{x})             & =\int\dt^3xe^{-\imagu\vec{p}\vec{x}}\int\frac{\dt^3q}{(2\pi)^3}\frac{1}{2\omega_{\vec{q}}}\big(a(\omega_{\vec{q}},\vec{q})e^{-\imagu(\omega_{\vec{q}}t-\vec{q}\vec{x})}+b(\omega_{\vec{q}},\vec{q})e^{\imagu(\omega_{\vec{q}}t-\vec{q}\vec{x})}\big) \\
                                                                                                      & =\frac{1}{2\omega_{\vec{p}}}\big(a(\omega_{\vec{p}},\vec{p})e^{-\imagu\omega_{\vec{p}}t}+b(\omega_{\vec{p}},-\vec{p})e^{\imagu\omega_{\vec{p}}t}\big)                                                                                                \\
        \dot{\tilde{\phi}}^{(3)}(t,\vec{p})=\int\dt^3x e^{-\imagu\vec{p}\vec{x}}\dot{\phi}(t,\vec{x}) & =\frac{1}{2}\big(-\imagu a(\omega_{\vec{p}},\vec{p})e^{-\imagu\omega_{\vec{p}}t}+\imagu b(\omega_{\vec{p}},-\vec{p})e^{\imagu\omega_{\vec{p}}t}\big)
    \end{align}
    we can find the inversion of the parametrization
    \begin{align}
        a(\omega_{\vec{p}},\vec{p})e^{-\imagu\omega_{\vec{p}}t} & =\int\dt^3x\big(\omega_{\vec{p}}\phi(t,\vec{x})+\imagu\dot{\phi}(t,\vec{x})\big)e^{-\imagu\vec{p}\vec{x}}                                                                                                                                          \\
                                                                & =\imagu e^{-\imagu\omega_{\vec{p}}t}\int\dt^3x\big(\phi(t,\vec{x})\overset{\leftarrow}{\partial_t}e^{\imagu(\omega_{\vec{p}}t-\vec{p}\vec{x})}-\phi(t,\vec{x})\overset{\rightarrow}{\partial_t}e^{\imagu(\omega_{\vec{p}}t-\vec{p}\vec{x})}\big)   \\
        b(\omega_{\vec{p}},-\vec{p})e^{\imagu\omega_{\vec{p}}t} & =\int\dt^3x\big(\omega_{\vec{p}}\phi(t,\vec{x})-\imagu\dot{\phi}(t,\vec{x})\big)e^{-\imagu\vec{p}\vec{x}}                                                                                                                                          \\
                                                                & =-\imagu e^{\imagu\omega_{\vec{p}}t}\int\dt^3x\big(\phi(t,\vec{x})\overset{\leftarrow}{\partial_t}e^{-\imagu(\omega_{\vec{p}}t+\vec{p}\vec{x})}-\phi(t,\vec{x})\overset{\rightarrow}{\partial_t}e^{-\imagu(\omega_{\vec{p}}t+\vec{p}\vec{x})}\big)
    \end{align}
\end{subequations}

Note how this has the form of the inner product defined in \ref{subsec:FourierDeformHypersurface}, to be precise
\begin{equation}
    a(\omega_{\vec{p}},\vec{p})=(\phi,e^{-\imagu(\omega_{\vec{p}}t-\vec{p}\vec{x})})_{\Sigma_t}\,,\qquad b(\omega_{\vec{p}},-\vec{p})=-(\phi,e^{\imagu(\omega_{\vec{p}}t-\vec{p}\vec{x})})_{\Sigma_t}
\end{equation}
and can therefore in principle be evaluated on any Cauchy surface.

It is now argued that the Fourier decomposition of a real source can be found by
\begin{subequations}
    \begin{align}
        \tilde{J}(\vec{p}) & =-\imagu e^{\imagu\omega_{\vec{p}}t}\big(\omega_{\vec{p}}\tilde{\phi}^{(3)}(t,\vec{p})+\imagu\dot{\tilde{\phi}}^{(3)}(t,\vec{p})\big)                                                                         \\
        \intertext{from which by substitution we derive the form}
                           & =-\imagu e^{\imagu\omega_{\vec{p}}t}\int\dt^3x\big(\omega_{\vec{p}}\phi(t,\vec{x})+\imagu\dot{\phi}(t,\vec{x})\big)e^{-\imagu\vec{p}\vec{x}}                                                                  \\
                           & =\int\dt^3x\big(\phi(t,\vec{x})\overset{\leftarrow}{\partial_t}e^{\imagu(\omega_{\vec{p}}t-\vec{p}\vec{x})}-\phi(t,\vec{x})\overset{\rightarrow}{\partial_t}e^{\imagu(\omega_{\vec{p}}t-\vec{p}\vec{x})}\big)
    \end{align}
\end{subequations}
which, by the same argument as before, is independent of the choice of Cauchy surface. The calculation is the same as before, repeated here for my own clarity (note $\overset{\leftrightarrow}{\partial}=\overset{\leftarrow}{\partial}-\overset{\rightarrow}{\partial}$, unlike before) \textcolor{green}{\textbf{and some errors corrected}}
\begin{subequations}
    \begin{align}
        \tilde{J}(\vec{p}) & =\int_{-\infty}^\infty\dt\eta\int_0^{2\pi}\dt\varphi\int_0^\pi\dt\alpha\tau r\Bigg[\phi(\tau,r)\big(r^\prime\overset{\leftrightarrow}{\partial_\tau}+\tau^\prime\overset{\leftrightarrow}{\partial_r}\big)e^{\imagu(\tau \omega_\perp\cosh(\eta-\eta_p)-r p_\perp\cos(\varphi-\varphi_p))}\Bigg] \\
                           & =\int_{-\infty}^\infty\dt\eta\int_0^{2\pi}\dt\varphi\int_0^\pi\dt\alpha\tau r\Bigg[\phi(\tau,r)\big(r^\prime\overset{\leftrightarrow}{\partial_\tau}+\tau^\prime\overset{\leftrightarrow}{\partial_r}\big)e^{\imagu(\tau \omega_\perp\cosh\eta-r p_\perp\cos\varphi)}\Bigg]                      \\
                           & =\int_0^\pi\dt\alpha\tau r\Bigg[\phi(\tau,r)(r^\prime\overset{\leftrightarrow}{\partial_\tau}+\tau^\prime\overset{\leftrightarrow}{\partial_r})\Big[J_0(r p_\perp)\times\big(-Y_0(\tau\omega_\perp)+\imagu J_0(\tau\omega_\perp)\big)\Big]\Bigg]                                                 \\
                           & =2\pi^2\int_0^\pi\dt\alpha\tau r\Bigg[(r^\prime\partial_\tau+\tau^\prime\partial_r)\phi(\tau,r)\Big[J_0(r p_\perp)\times\big(-Y_0(\tau\omega_\perp)+\imagu J_0(\tau\omega_\perp)\big)\Big]+\nonumber                                                                                             \\
                           & \phantom{=}\qquad + \phi(\tau,r)\Big[\tau^\prime\times p_\perp J_1(r p_\perp)\times\big(-Y_0(\tau\omega_\perp)+\imagu J_0(\tau\omega_\perp)\big)+\nonumber                                                                                                                                       \\
                           & \phantom{=}\qquad\phantom{+\phi(\tau,r)\Big[}+r^\prime\times J_0(r p_\perp)\times\omega_\perp\big(-Y_1(\tau\omega_\perp)+\imagu J_1(\tau\omega_\perp)\big)\Big]\Bigg]
    \end{align}
\end{subequations}

\paragraph*{Notational Simplification}\mbox{}\\

For numerical ease and clarity, define helper functions
\begin{subequations}
    \begin{align}
        H_1 & =\Big[J_0(r p_\perp)\times\big(-Y_0(\tau\omega_\perp)+\imagu J_0(\tau\omega_\perp)\big)\Big]                                       \\
        H_2 & =\Big[\tau^\prime\times p_\perp J_1(r p_\perp)\times\big(-Y_0(\tau\omega_\perp)+\imagu J_0(\tau\omega_\perp)\big)+\nonumber        \\
            & \phantom{=\Big[}+r^\prime\times J_0(r p_\perp)\times\omega_\perp\big(-Y_1(\tau\omega_\perp)+\imagu J_1(\tau\omega_\perp)\big)\Big]
    \end{align}
\end{subequations}
What we are interested in is not the field $\phi$, but rather the pionic component $\pi=\sqrt{2}\Im\phi$. We can simply replace all $\phi$'s with $\pi$'s since the \textbf{homogeneous} KG equation holds separately for imaginary and real part.

In the spirit of identifying the pion field with fluid variables, one could argue
\begin{equation}
    \partial_\mu\phi=\partial_\mu(\rho\exp(\imagu\vartheta))=\imagu\phi\partial_\mu\vartheta=\imagu\phi\chi u_\mu
\end{equation}
Applied separately to real and imaginary part we get
\begin{equation}
    \partial_\mu\pi=\sqrt{2}\Im\partial_\mu\phi=\sqrt{2}\chi u_\mu\Re\phi=\sigma\chi u_\mu\,,\qquad\partial_\mu\sigma=\sqrt{2}\Re\partial_\mu\phi=-\sqrt{2}\chi u_\mu\Im\phi=-\pi\chi u_\mu
\end{equation}
Equation \eqref{eq:FluidsToFields_complexPhase} has an integration constant $\vartheta_0$. Compared to the fields $\pi_0,\sigma_0$ with the choice $\vartheta_0=0$, the pion field for any integration constant can be computed by
\begin{equation}
    \pi(\vartheta_0)=\Im(\phi e^{\imagu\vartheta_0})=\sigma_0\sin\vartheta_0+\pi_0\cos\vartheta_0
\end{equation}

It is thus sensible to compute the following contributions separately
\begin{subequations}
    \begin{align}
        C_{1\sigma} & =2\pi^2\int_0^\pi\dt\alpha\tau r\;(r^\prime\partial_\tau+\tau^\prime\partial_r)\sigma_0\times H_1            \\
                    & =2\pi^2\int_0^\pi\dt\alpha\tau r\;\cdot (-1)\cdot\pi_0\chi (r^\prime u_\tau+\tau^\prime\partial_r)\times H_1 \\
        C_{1\pi}    & =2\pi^2\int_0^\pi\dt\alpha\tau r\;(r^\prime\partial_\tau+\tau^\prime\partial_r)\pi_0\times H_1               \\
                    & =2\pi^2\int_0^\pi\dt\alpha\tau r\;\sigma_0\chi (r^\prime u_\tau+\tau^\prime\partial_r)\times H_1             \\
        C_{2\sigma} & =2\pi^2\int_0^\pi\dt\alpha\tau r\;\sigma_0\times H_2                                                         \\
        C_{2\pi}    & =2\pi^2\int_0^\pi\dt\alpha\tau r\;\pi_0\times H_2
    \end{align}
\end{subequations}
and the final result via
\begin{equation}
    \tilde{J}(\vec{p})=(C_{1\sigma}+C_{2\sigma})\sin\vartheta_0+(C_{1\pi}+C_{2\pi})\cos\vartheta_0
\end{equation}