\chapter{Introduction}

The quark-gluon-plasma (QGP) is an exotic phase \todo{at high temperatures?} of quantum chromodynamics (QCD), the gauge theory describing strong interaction between quarks and gluons in the standard model. Collider experiments, as they are performed for example at the Large Hadron Collider (LHC), enable interesting experiments in order to probe this regime of the QCD phase diagram. In Heavy Ion Collisions (HIC) heavy nuclei are accelerated to ultra-relativistic energies in order to produce droplets of QGP at the moment of collision. At times a few ${\text{fm}/c}$ after the collision, this "fireball" of QGP can be understood mainly as a relativistic liquid that subsequently expands and cools down. As the number of interactions of elementary particles decreases, the system transitions from a liquid state to a state of non-interacting particles, that can be detected, counted and classified w.r.t. particle species and momentum distribution in the surrounding detectors.

Whereas the resulting spectra of final state particles are generally described very well in the fluid picture, experiments \cite{ALICECollaboration_2022} and fluid models show a discrepancy in the low-$p_T$ (transverse momentum) regime \cite{KirchnerEtAl_2023}. The possibility of a disoriented chiral condensate (DCC) as a macroscopically excited field mode was proposed \cite{Bjorken_1997, Amelino-CameliaEtAl_1997, MohantySerreau_2005}, since such a close-to-classical field would not be captured by the thermal distribution functions usually assumed in fluid models. Applying semi-classical field theory, we try to analyze the possibility of a pion condensate and how it would improve upon the largely succesful methods using the fluid picture of HIC.

\section{Conventions}

In this thesis, we choose the mostly plus convention for the Minkowski metric,
\begin{equation}
    \eta=\text{diag}(-1,+1,+1,+1)\,.
\end{equation}

The Fourier transform $\tilde{f}(p)$ or plane wave decomposition of a function $f(x)$ is defined by
\begin{equation}
    f(x)=\int\frac{\dt^4p}{(2\pi)^4}\tilde{f}(p)e^{\imagu px}\,,\qquad \tilde{f}(p)=\int\dt^4x f(x)e^{-\imagu px}\,.
    \label{eq:FourierConvention}
\end{equation}