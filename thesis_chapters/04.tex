\chapter{Particle Spectra from Classical Source}

\section{Canonical Quantization}

\subsection{Real Scalar Field}

Consider a real scalar field $\phi$ with Lagrangian density ($\eta=\text{diag}(-,+,+,+)$)
\begin{equation}
    \mathcal{L}=-\frac{1}{2}(\partial_\mu\phi)(\partial^\mu\phi)-V(\phi)=\frac{1}{2}(\dot{\phi}^2-\vec{\nabla}^2\phi)-V(\phi)
    \label{eq:LagrangianRealScalar}
\end{equation}
with associated Hamiltonian density
\begin{equation}
    \mathcal{H}=\pi\dot{\phi}-\mathcal{L}=\frac{1}{2}(\pi^2+(\vec{\nabla}\phi)^2)+V(\phi)
\end{equation}
where $\pi=\frac{\partial\mathcal{L}}{\partial\dot{\phi}}=\dot{\phi}$. Choose the free scalar field, $V(\phi)=\frac{1}{2}m^2\phi^2$. The equations of motion arising from this is the Klein-Gordon equation
\begin{equation}
    (\partial_\mu\partial^\mu-m^2)\phi(t,\vec{x})=0\,.
    \label{eq:KGEq}
\end{equation}

The equations of motion \eqref{eq:KGEq} have the general solution
\begin{subequations}
    \begin{align}
        \phi(t,x)               & =\int\frac{\dt ^3p}{(2\pi)^3}\mathcal{N}_{\vec{p}}\left\{a_{\vec{p}}e^{-\imagu(\omega_{\vec{p}}t-\vec{p}\vec{x})}+b_{\vec{p}}^* e^{\imagu(\omega_{\vec{p}}t-\vec{p}\vec{x})}\right\}                                                 \\
        (\implies)\quad\pi(t,x) & =\int\frac{\dt ^3p}{(2\pi)^3}\mathcal{N}_{\vec{p}}\left\{-\imagu\omega_{\vec{p}}a_{\vec{p}}e^{-\imagu(\omega_{\vec{p}}t-\vec{p}\vec{x})}+\imagu\omega_{\vec{p}}b_{\vec{p}}^* e^{\imagu(\omega_{\vec{p}}t-\vec{p}\vec{x})}\right\}\,.
    \end{align}
\end{subequations}
only subject to the condition $\omega_{\vec{p}}=\sqrt{m^2+p^2}$. $a_{\vec{p}}$ and $b_{\vec{p}}^*$ are complex Fourier coefficients. Reality of $\phi(t,x)$ further implies $a_{\vec{p}}=b_{\vec{p}}$. The normalization is typically chosen as $\mathcal{N}_{\vec{p}}^2\omega_{\vec{p}}=\frac{1}{2}$ for reasons that will become clear in a moment. If one uses
\begin{defin}[Poisson Brackets on Field Space]{def:PoissonBrack}
    \begin{equation}
        \poisson{A,B}=\int\dt^3x\left[\frac{\delta A}{\delta\phi}\frac{\delta B}{\delta\pi}-\frac{\delta A}{\delta\pi}\frac{\delta B}{\delta\phi}\right]
    \end{equation}
\end{defin}
the field and momentum fields satisfy
\begin{equation}
    \poisson{\phi(t,x),\phi(t,y)}=\poisson{\pi(t,x),\pi(t,y)}=0\,,\quad\poisson{\phi(t,x),\pi(t,y)}=\delta^{(d)}(x-y)\,.
\end{equation}
Quantization is achieved by the replacement
\begin{equation}
    \imagu\{\cdot,\cdot\}\to\commut{\cdot,\cdot}\,,
\end{equation}
lifting fields to operators, $\phi\to\hat{\phi}$ and $\pi\to\hat{\pi}$, and therefore also $a_{\vec{p}}\to\hat{a}_{\vec{p}}$ and $a_{\vec{p}}^\dagger\to\hat{a}_{\vec{p}}^\dagger$ (though the $\hat{\cdot}$ will be omitted). The fundamental commutator $\commut{\phi(t,x),\pi(t,y)}=\imagu\delta^{(d)}(x-y)$ then implies
\begin{impt}[Commutators of $a_{\vec{p}}$, $a_{\vec{q}}^\dagger$]{impt:LadderCommut}
    \begin{equation}
        \commut{a_{\vec{p}},a_{\vec{q}}}=\commut{a_{\vec{p}}^\dagger,a_{\vec{q}}^\dagger}=0\,,\quad\commut{a_{\vec{p}},a_{\vec{q}}^\dagger}=(2\pi)^3\delta^{(3)}(\vec{p}-\vec{q})\,.
    \end{equation}
\end{impt}

\begin{calc}[Commutators of $a_{\vec{p}}$, $a_{\vec{q}}^\dagger$]{calc:LadderCommut}
    Notice the relations
    \begin{subequations}
        \begin{align}
            a_{\vec{p}} & =\frac{1}{2\mathcal{N}_{\vec{p}}}\int\dt ^3x\left\{\phi(t,x)+\frac{\imagu}{\omega_{\vec{p}}}\pi(t,x)\right\}e^{\imagu(\omega_{\vec{p}}t-\vec{p}\vec{x})}  \\
            a_{\vec{p}}^\dagger       & =\frac{1}{2\mathcal{N}_{\vec{p}}}\int\dt ^3x\left\{\phi(t,x)-\frac{\imagu}{\omega_{\vec{p}}}\pi(t,x)\right\}e^{-\imagu(\omega_{\vec{p}}t-\vec{p}\vec{x})}
        \end{align}
        \label{eq:AnnCrePhiPi_Relation}
    \end{subequations}
    The non-vanishing commutator is derived as follows:
    \begin{align*}
        \commut{a_{\vec{p}},a_{\vec{q}}^\dagger} & =\frac{1}{4\mathcal{N}_{\vec{p}}\mathcal{N}_{\vec{q}}}\int\dt^3x\dt^3y\left\{-\frac{\imagu}{\omega_{\vec{q}}}\commut{\phi(t,x),\pi(t,y)}e^{\imagu\left((\omega_{\vec{p}}-\omega_{\vec{q}})t-(\vec{p}\vec{x}-\vec{q}\vec{y})\right)}\right. \\
                                                 & \phantom{=}\quad\left.                     +\frac{\imagu}{\omega_{\vec{p}}}\commut{\pi(t,x),\phi(t,y)}e^{-\imagu\left((\omega_{\vec{p}}-\omega_{\vec{q}})t-(\vec{p}\vec{x}-\vec{q}\vec{y})\right)}\right\}                                 \\
                                                 & =\frac{1}{4\mathcal{N}_{\vec{p}}\mathcal{N}_{\vec{q}}}\int\dt^3x\left\{\frac{1}{\omega_{\vec{q}}}e^{\imagu\left((\omega_{\vec{p}}-\omega_{\vec{q}})t-(\vec{p}-\vec{q})x\right)}\right.                                                     \\
                                                 & \phantom{=}\quad\left.                     +\frac{1}{\omega_{\vec{p}}}e^{-\imagu\left((\omega_{\vec{p}}-\omega_{\vec{q}})t-(\vec{p}-\vec{q})x\right)}\right\}                                                                              \\
                                                 & =\frac{(2\pi)^3}{2\mathcal{N}_{\vec{p}}^2\omega_{\vec{p}}}\delta^{(3)}(\vec{p}-\vec{q})
    \end{align*}
    whereas the vanishing commutators are calculated as
    \begin{align*}
        \commut{a_{\vec{p}},a_{\vec{q}}} & =\frac{1}{4\mathcal{N}_{\vec{p}}\mathcal{N}_{\vec{q}}}\int\dt^3x\dt^3y\left\{\frac{\imagu}{\omega_{\vec{q}}}\commut{\phi(t,x),\pi(t,y)}e^{\imagu\left((\omega_{\vec{p}}+\omega_{\vec{q}})t-(\vec{p}\vec{x}+\vec{q}\vec{y})\right)}\right. \\
                                         & \phantom{=}\quad\left.                     +\frac{\imagu}{\omega_{\vec{p}}}\commut{\pi(t,x),\phi(t,y)}e^{\imagu\left((\omega_{\vec{p}}+\omega_{\vec{q}})t-(\vec{p}\vec{x}+\vec{q}\vec{y})\right)}\right\}                                 \\
                                         & =0
    \end{align*}
    and simmilarly for $\commut{a_{\vec{p}}^\dagger,a_{\vec{q}}^\dagger}=0$.
\end{calc}

After this quantization, the fields are written as
\begin{subequations}
    \begin{align}
        \phi(t,x)               & =\int\frac{\dt ^3p}{(2\pi)^3}\frac{1}{\sqrt{2\omega_{\vec{p}}}}\left\{a_{\vec{p}}e^{-\imagu(\omega_{\vec{p}}t-\vec{p}\vec{x})}+a_{\vec{p}}^\dagger e^{\imagu(\omega_{\vec{p}}t-\vec{p}\vec{x})}\right\}                                                 \label{eq:CanonicalQuant_RealScalar}\\
       \pi(t,x) & =\int\frac{\dt ^3p}{(2\pi)^3}\frac{\imagu\omega_{\vec{p}}}{\sqrt{2\omega_{\vec{p}}}}\left\{-a_{\vec{p}}e^{-\imagu(\omega_{\vec{p}}t-\vec{p}\vec{x})}+a_{\vec{p}}^\dagger e^{\imagu(\omega_{\vec{p}}t-\vec{p}\vec{x})}\right\}\,.
    \end{align}
\end{subequations}

To express the Hamiltonian in terms of $a_{\vec{p}}$ and $a_{\vec{p}}^\dagger$ rewrite
\begin{subequations}
    \begin{align}
        \phi(t,x)               & =\int\frac{\dt ^3p}{(2\pi)^3}\frac{1}{\sqrt{2\omega_{\vec{p}}}}\left\{a_{\vec{p}}e^{-\imagu\omega_{\vec{p}}t}+a_{-\vec{p}}^\dagger e^{\imagu\omega_{\vec{p}}t}\right\}e^{\imagu\vec{p}\vec{x}}                                             \\
        \pi(t,x) & =\int\frac{\dt ^3p}{(2\pi)^3}\frac{\imagu\omega_{\vec{p}}}{\sqrt{2\omega_{\vec{p}}}}\left\{-a_{\vec{p}}e^{-\imagu\omega_{\vec{p}}t}+a_{-\vec{p}}^\dagger e^{\imagu\omega_{\vec{p}}t}\right\}e^{\imagu\vec{p}\vec{x}}\,.
    \end{align}
\end{subequations}
Omit the time dependence for the next calculation, for example by choosing $t=0$. The Hamiltonian is now easily computed to be
\begin{subequations}
    \begin{align}
        H&=\frac{1}{2}\int\dt^3x\frac{\dt^3p}{(2\pi)^3}\frac{\dt^3q}{(2\pi)^3}\frac{1}{\sqrt{4\omega_{\vec{p}}\omega_{\vec{q}}}}e^{\imagu(\vec{p}+\vec{q})\vec{x}}\Big[-\omega_{\vec{p}}\omega_{\vec{q}}(-a_{\vec{p}}+a_{-\vec{p}}^\dagger)(-a_{\vec{q}}+a_{-\vec{q}}^\dagger)+\nonumber\\
        &\phantom{=}\qquad+(-\vec{p}\vec{q}+m^2)(a_{\vec{p}}+a_{-\vec{p}}^\dagger)(a_{\vec{q}}+a_{-\vec{q}}^\dagger)\Big]\\
        &=\frac{1}{2}\int\frac{\dt^3p}{(2\pi)^3}\frac{1}{2\omega_{\vec{p}}}\Big[-\omega_{\vec{p}}^2(\cancel{a_{\vec{p}}a_{-\vec{p}}}-a_{-\vec{p}}^\dagger a_{-\vec{p}}-a_{\vec{p}}a_{\vec{p}}^\dagger+\bcancel{a_{-\vec{p}}^\dagger a_{\vec{p}}^\dagger})+\nonumber\\
        &\phantom{=}\qquad+(\vec{p}^2+m^2)(\cancel{a_{\vec{p}}a_{-\vec{p}}}+a_{-\vec{p}}^\dagger a_{-\vec{p}}+a_{\vec{p}}a_{\vec{p}}^\dagger+\bcancel{a_{-\vec{p}}^\dagger a_{\vec{p}}^\dagger})\Big]\\
        &=\int\frac{\dt^3p}{(2\pi)^3}\frac{\omega_{\vec{p}}}{2}(a_{\vec{p}}^\dagger a_{\vec{p}}+a_{\vec{p}}a_{\vec{p}}^\dagger)\\
        &=\int\frac{\dt^3p}{(2\pi)^3}\omega_{\vec{p}}(a_{\vec{p}}^\dagger a_{\vec{p}}+\frac{1}{2}[a_{\vec{p}},a_{\vec{p}}^\dagger])
    \end{align}
\end{subequations}
Since only the combination $a_{\vec{p}}a_{\vec{p}}^\dagger$ shows up, the explicit time dependence would have dropped out anyways. The commutation relation between $H$, $a_{\vec{p}}$ and $a_{\vec{p}}^\dagger$ are given by
\begin{equation}
    [H,a_{\vec{p}}]=-\omega_{\vec{p}}a_{\vec{p}}\,,\qquad[H,a_{\vec{p}}^\dagger]=\omega_{\vec{p}}a_{\vec{p}}^\dagger
\end{equation}
From quantum mechanics it is now clear that $H$ every momentum mode $\vec{p}$ has a discrete spectrum of excitations or energy eigenstates, such that
\begin{equation}
    H\ket{n_{\vec{p}}}=(\omega_{\vec{p}}+E_0)\ket{n_{\vec{p}}}
\end{equation}
and the operator $a_{\vec{p}}$ ($a_{\vec{p}}^\dagger$) annihilates (creates) excitations,
\begin{equation}
    a_{\vec{p}}\ket{n_{\vec{p}}}=\sqrt{n_{\vec{p}}} \ket{(n-1)_{\vec{p}}}\,,\qquad a_{\vec{p}}^\dagger\ket{n_{\vec{p}}}= \sqrt{(n+1)_{\vec{p}}}\ket{(n+1)_{\vec{p}}}
\end{equation}
$E_0$ is the (IR) divergent vacuum energy. The vacuum is defined by $a_{\vec{p}}\ket{0}=0$ $\forall\vec{p}$. The operator
\begin{equation}
    N_{\vec{p}}=a_{\vec{p}}^\dagger a_{\vec{p}}
\end{equation}
is the number operator for a given momentum mode and its expectation value $n(\vec{p})=\langle N_{\vec{p}}\rangle$ has the interpretation of the momentum space number density,
\begin{equation}
    N=\int\frac{\dt^3p}{(2\pi)^3}n(\vec{p})\,,\qquad n(\vec{p})=(2\pi)^3\frac{\dt N}{\dt^3p}
\end{equation}

\subsection{Complex Scalar Field}

Consider a complex scalar field $\phi=\frac{1}{\sqrt{2}}(\phi_1+\imagu\phi_2)$ with $\phi_k,k\in\{1,2\}$ two real scalar fields. The Lagrangian of $\phi$ can be written as the sum of the Lagrangians $\mathcal{L}_k$ of $\phi_k$. Similarly for the Hamiltonian
\begin{equation}
    \mathcal{L}=-(\partial_\mu\phi^*)(\partial^\mu\phi)-m^2\phi\phi^*=\sum_{k}\Big\{-\frac{1}{2}\Big((\partial_\mu\phi_k)(\partial^\mu\phi_k)+m^2\phi_k^2\Big)\Big\}=\sum_k\mathcal{L}_k
\end{equation}
With the conjugate momenta $\pi_k=\dot{\phi}_k$ the conjugate momentum of $\phi$ turns out to be
\begin{equation}
    \pi=\frac{\partial\mathcal{L}}{\partial\dot{\phi}}=\dot{\phi}^*=\frac{\pi_1-\imagu\pi_2}{\sqrt{2}}
\end{equation}
and the Hamiltonian is therefore
\begin{equation}
    \mathcal{H}=\pi\dot{\phi}+\pi^*\dot{\phi}^*-\mathcal{L}=\dot{\phi}\dot{\phi}^*+(\vec{\nabla}\phi)(\vec{\nabla}\phi^*)+m^2\phi\phi^*=\sum_k\frac{1}{2}(\pi_k^2+(\vec{\nabla}\phi_k)^2+m^2\phi_k^2)=\sum_k\mathcal{H}_k
\end{equation}

Quantization rules are imposed as before on the real scalar field $\phi_k$. From this, it is immediately clear that $\phi(t,\vec{x})$ takes the form
\begin{subequations}    
    \begin{align}
        \phi(t,\vec{x})&=\int\frac{\dt^3p}{(2\pi)^3}\frac{1}{\sqrt{2\omega_{\vec{p}}}}\Big\{\frac{a_{\vec{p},(1)}+\imagu a_{\vec{p},(2)}}{\sqrt{2}}e^{-\imagu(\omega_{\vec{p}}t-\vec{p}\vec{x})}+\frac{a_{\vec{p},(1)}^\dagger+\imagu a_{\vec{p},(2)}^\dagger}{\sqrt{2}}e^{\imagu(\omega_{\vec{p}}t-\vec{p}\vec{x})}\Big\}\\
        \pi(t,\vec{x})&=\int\frac{\dt^3p}{(2\pi)^3}\frac{\imagu\omega_{\vec{p}}}{\sqrt{2\omega_{\vec{p}}}}\Big\{-\frac{a_{\vec{p},(1)}-\imagu a_{\vec{p},(2)}}{\sqrt{2}}e^{-\imagu(\omega_{\vec{p}}t-\vec{p}\vec{x})}+\frac{a_{\vec{p},(1)}^\dagger-\imagu a_{\vec{p},(2)}^\dagger}{\sqrt{2}}e^{\imagu(\omega_{\vec{p}}t-\vec{p}\vec{x})}\Big\}
    \end{align}
\end{subequations}
    It is intuitive to define
    \begin{equation}
    a_{\vec{p}}=\frac{a_{\vec{p},(1)}+\imagu a_{\vec{p},(2)}}{\sqrt{2}}\,,\qquad b_{\vec{p}}^\dagger=\frac{a_{\vec{p},(1)}^\dagger+\imagu a_{\vec{p},(2)}^\dagger}{\sqrt{2}}
\end{equation}
recovering the looking familiar expression
\begin{subequations}    
    \begin{align}
        \phi(t,\vec{x})&=\int\frac{\dt^3p}{(2\pi)^3}\frac{1}{\sqrt{2\omega_{\vec{p}}}}\Big\{a_{\vec{p}}e^{-\imagu(\omega_{\vec{p}}t-\vec{p}\vec{x})}+b_{\vec{p}}^\dagger e^{\imagu(\omega_{\vec{p}}t-\vec{p}\vec{x})}\Big\}\\
        \pi(t,\vec{x})&=\int\frac{\dt^3p}{(2\pi)^3}\frac{\imagu\omega_{\vec{p}}}{\sqrt{2\omega_{\vec{p}}}}\Big\{-b_{\vec{p}}e^{-\imagu(\omega_{\vec{p}}t-\vec{p}\vec{x})}+a_{\vec{p}}^\dagger e^{\imagu(\omega_{\vec{p}}t-\vec{p}\vec{x})}\Big\}
    \end{align}
\end{subequations}
where now, unlike before, explicitly $a_{\vec{p}}\neq b_{\vec{p}}$ is found.

The commutation relations of $a_{\vec{p},(k)}$, $a_{\vec{p},(k)}^\dagger$ are trivially given by
\begin{equation}
    [a_{\vec{p},(j)},a_{\vec{q},(k)}^\dagger]=(2\pi)^3\delta^{(3)}(\vec{p}-\vec{q})\delta_{jk}\,,\qquad[a_{\vec{p},(j)}^{(\dagger)},a_{\vec{p},(k)}^{(\dagger)}]=0
\end{equation}
and lead to
\begin{subequations}
    \begin{align}
        [a_{\vec{p}},b_{\vec{q}}]&=\frac{1}{2}[a_{\vec{p},(1)}+\imagu a_{\vec{p},(2)},a_{\vec{q},(1)}-\imagu a_{\vec{q},(2)}]=0\\
        [a_{\vec{p}}^{(\dagger)},a_{\vec{q}}^{(\dagger)}]&=0\\
        [a_{\vec{p}},b_{\vec{q}}^\dagger]&=\frac{1}{2}[a_{\vec{p},(1)}+\imagu a_{\vec{p},(2)},a_{\vec{q},(1)}^\dagger+\imagu a_{\vec{q},(2)}^\dagger]=\frac{1}{2}\big([a_{\vec{p},(1)},a_{\vec{q},(1)}^\dagger]-[a_{\vec{p},(2)},a_{\vec{q},(2)}^\dagger]\big)=0\\
        [a_{\vec{p}},a_{\vec{q}}^\dagger]&=\frac{1}{2}[a_{\vec{p},(1)}+\imagu a_{\vec{p},(2)},a_{\vec{q},(1)}^\dagger-\imagu a_{\vec{q},(2)}^\dagger]=\frac{1}{2}\big([a_{\vec{p},(1)},a_{\vec{q},(1)}^\dagger]+[a_{\vec{p},(2)},a_{\vec{q},(2)}^\dagger]\big)=(2\pi)^3\delta^{(3)}(\vec{p}-\vec{q})\\
        [b_{\vec{p}},b_{\vec{q}}^\dagger]&=\frac{1}{2}[a_{\vec{p},(1)}-\imagu a_{\vec{p},(2)},a_{\vec{q},(1)}^\dagger+\imagu a_{\vec{q},(2)}^\dagger]=\frac{1}{2}\big([a_{\vec{p},(1)},a_{\vec{q},(1)}^\dagger]+[a_{\vec{p},(2)},a_{\vec{q},(2)}^\dagger]\big)=(2\pi)^3\delta^{(3)}(\vec{p}-\vec{q})
    \end{align}
\end{subequations}

From this, again the Hamiltonian is derived to be
\begin{subequations}
    \begin{align}        
        H&=\int\frac{\dt^3p}{(2\pi)^3}\omega_{\vec{p}}\Big(a_{\vec{p},(1)}^\dagger a_{\vec{p},(1)}+a_{\vec{p},(2)}^\dagger a_{\vec{p},(2)}+\frac{1}{2}\big([a_{\vec{p},(1)},a_{\vec{p},(1)}^\dagger]+[a_{\vec{p},(2)},a_{\vec{p},(2)}^\dagger]\big)\Big)\\
        &=\int\frac{\dt^3p}{(2\pi)^3}\omega_{\vec{p}}\Big(a_{\vec{p}}^\dagger a_{\vec{p}}+b_{\vec{p}}^\dagger b_{\vec{p}}+\frac{1}{2}\big([a_{\vec{p}},a_{\vec{p}}^\dagger]+[b_{\vec{p}},b_{\vec{p}}^\dagger]\big)\Big)
    \end{align}
\end{subequations}
using $a_{\vec{p}}^\dagger a_{\vec{p}}=\frac{1}{2}(a_{\vec{p},(1)}^\dagger-\imagu a_{\vec{p},(2)}^\dagger)(a_{\vec{p},(1)}+\imagu a_{\vec{p},(2)})$ and $b_{\vec{p}}^\dagger b_{\vec{p}}=\frac{1}{2}(a_{\vec{p},(1)}^\dagger+\imagu a_{\vec{p},(2)}^\dagger)(a_{\vec{p},(1)}-\imagu a_{\vec{p},(2)})$. Whereas $n_{\vec{p}}=\langle a_{\vec{p}}^\dagger a_{\vec{p}}\rangle$ has the interpretation of a particle number density, $\overline{n}_{\vec{p},J}=\langle b_{\vec{p}}^\dagger b_{\vec{p}}\rangle$ is understood as the antiparticle density.

\section{Particle Spectra from Classical Sources}

\subsection{Real Scalar Field}

Follow \textbf{CITE: REINHARD: FIELD QUANTIZATION} around eq. 4.140. Define the Pauli-Jordan function $\Delta(x)$ via
\begin{align}
    \imagu\Delta(x-y)&=\int\frac{\dt^3p}{(2\pi)^3}\frac{1}{2\omega_{\vec{p}}}\Big(e^{-\imagu (\omega_{\vec{p}}(x^0-y^0)-\vec{p}(\vec{x}-\vec{y}))}-e^{\imagu (\omega_{\vec{p}}(x^0-y^0)-\vec{p}(\vec{x}-\vec{y}))}\Big)\\
    &=\int\frac{\dt^3p}{(2\pi)^3}\frac{1}{2\omega_{\vec{p}}}\Big(e^{\imagu p(x-y)}-e^{-\imagu p(x-y)}\Big)\\
    &=\int\frac{\dt^3p}{(2\pi)^3}\frac{1}{2\omega_{\vec{p}}}\Big(e^{-\imagu \omega_{\vec{p}}(x^0-y^0)}-e^{\imagu \omega_{\vec{p}}(x^0-y^0)}\Big)e^{\imagu\vec{p}(\vec{x}-\vec{y})}\\
    &=2\pi\int\frac{\dt^4p}{(2\pi)^4}\epsilon(p^0)\delta(p^2+m^2)e^{-\imagu(p^0(x^0-y^0)-\vec{p}(\vec{x}-\vec{y}))}\\
    &=2\pi\int\frac{\dt^4p}{(2\pi)^4}\epsilon(p^0)\delta(p^2+m^2)e^{\imagu p(x-y)}
\end{align}
which satisfies (...) $(\partial_\mu\partial^\mu-m^2)\Delta=0$. $\epsilon(x)$ is the sign function. The retarded and advanced propagators $\Delta_{R,A}(x)$ are given by 
\begin{equation}
    \Delta_R(x)=\Theta(x^0)\Delta(x)\,,\qquad\Delta_A(x)=\Theta(x^0)\Delta(x)\,.
\end{equation}
which immediately implies $\Delta(x)=\Delta_R(x)-\Delta_A(x)$. These function satisfy
\begin{equation}
    (\partial_\mu\partial^\mu-m^2)\Delta_{R,A}(x)=\delta^{(4)}(x)
\end{equation}

\begin{calc}[Greens Functions]{calc:GreensFunctions}
    Solve $(\partial_\mu\partial^\mu-m^2)D(x)=\delta^{(4)}(x)$. Using $$D(x)=\int\frac{\dt^4p}{(2\pi)^4}\tilde{D}(p)e^{\imagu px}$$ one finds
    \begin{align}
        \int\frac{\dt^4p}{(2\pi)^4}(-p^2-m^2)\tilde{D}(p)&=\delta^{(4)}(x)
        \tilde{D}(p)=-\frac{1}{p^2+m^2}
    \end{align}
    and thus
    \begin{align}
        D(x)&=\int\frac{\dt^4p}{(2\pi)^4}\frac{-1}{p^2+m^2}e^{-\imagu px}\\
        &=-\int\frac{\dt^3p}{(2\pi)^3}\frac{\dt p^0}{2\pi}\frac{1}{(p^0+\omega_{\vec{p}})(-p^0+\omega_{\vec{p}})}e^{\imagu(p^0t-\vec{p}\vec{x})}
    \end{align}

    \begin{defin}[Retarded Propagator, Contour]{def:RetPropCont}
        \centering
        \begin{tikzpicture}
            \draw (0,-1) -- (0,1);
            \draw (-3,0) -- (3,0);
    
            \coordinate[label=below:$-\omega_{\vec{p}}$] (n1) at (-2,0);
            \draw[fill] (n1) circle (0.05);
    
            \coordinate[label=below:$\omega_{\vec{p}}$] (n2) at (2,0);
            \draw[fill] (n2) circle (0.05);
    
            \draw[line width=1] (-2.5,0) arc (-180:0:0.5);
            \draw[line width=1,-{Latex[length=7,width=7,sep=2]}] (-3,0) -- (-2.5,0);
            \draw[line width=1] (-3,0) -- (-2.5,0);
            \draw[line width=1] (-1.5,0) -- (1.5,0);
            \draw[line width=1] (1.5,0) arc (-180:0:0.5);
            \draw[line width=1,-{Latex[length=7,width=7,sep=2]}] (2.5,0) -- (3,0);
            \draw[line width=1] (2.5,0) -- (3,0);
        \end{tikzpicture}
    \end{defin}

    If $t>0$, close the integration contour in the upper imaginary half plane.
    \begin{align}
        \int\frac{\dt p^0}{2\pi}\frac{1}{(p^0+\omega_{\vec{p}})(-p^0+\omega_{\vec{p}})}e^{\imagu p^0t}&=2\pi\imagu\Big(\lim_{p^0\to\omega_{\vec{p}}}\frac{1}{2\pi}(p^0-\omega_{\vec{p}})\frac{e^{\imagu p^0t}}{(p^0+\omega_{\vec{p}})(-p^0+\omega_{\vec{p}})}+\lim_{p^0\to-\omega_{\vec{p}}}\frac{1}{2\pi}(p^0+\omega_{\vec{p}})\frac{e^{\imagu p^0t}}{(p^0+\omega_{\vec{p}})(-p^0+\omega_{\vec{p}})}\Big)\\
        &=\imagu\Big(\frac{e^{-\imagu\omega_{\vec{p}}t}-e^{\imagu\omega_{\vec{p}}t}}{2\omega_{\vec{p}}}\Big)
    \end{align}
    For $t<0$, close the integration in the lower half plane, such that there is no residue within the integration contour. This leads to
    \begin{equation}
        D_R(x)=\frac{1}{\imagu}\Theta(t)\int\frac{\dt^3p}{(2\pi)^3}\frac{1}{2\omega_{\vec{p}}}(e^{-\imagu(\omega_{\vec{p}}t-\vec{p}\vec{x})}-e^{\imagu(\omega_{\vec{p}}t-\vec{p}\vec{x})})=-\imagu\Theta(x^0)\int\frac{\dt^3p}{(2\pi)^3}\frac{1}{2\omega_{\vec{p}}}(e^{\imagu px}-e^{-\imagu px})=\Theta(x^0)\Delta(x)\equiv\Delta_R(x)
    \end{equation}
\end{calc}

Consider now a real scalar field that evolves according to the inhomogeneous Klein-Gordon equation
\begin{equation}
    (\partial_\mu\partial^\mu-m^2)\phi=-J
\end{equation}
The solution is constructed by superposition of homogeneous solutions and a particular inhomogeneous solutions. Requesting $\phi\equiv 0$ for vanishing source, one finds
\begin{subequations}
    \begin{align}        
        \phi_J(x)&=-\int\dt^4y\Delta_R(x-y)J(y)\\
        &=\imagu\int\dt^4y\Theta(x^0-y^0)\int\frac{\dt^3p}{(2\pi)^3}\frac{1}{2\omega_{\vec{p}}}(e^{\imagu p(x-y)}-e^{-\imagu p(x-y)})J(y)\\
        \overset{x^0\gg y^0}&{=}\imagu \int\frac{\dt^3p}{(2\pi)^3}\frac{1}{2\omega_{\vec{p}}}(J(p)e^{\imagu px}-J(-p)e^{-\imagu px})\\
        &=\imagu\int\frac{\dt^3p}{(2\pi)^3}\frac{1}{2\omega_{\vec{p}}}(J(p)e^{-\imagu(\omega_{\vec{p}}t-\vec{p}\vec{x})}-J(-p)e^{\imagu(\omega_{\vec{p}}t-\vec{p}\vec{x})})\label{eq:ClassicalFieldAfterSource}
    \end{align}
\end{subequations}
where $J(p)=\int\dt^4y J(x)e^{-\imagu py}$ was used.

Taking the homogeneous solution $\phi_0$ as given by \eqref{eq:CanonicalQuant_RealScalar} into account, the field after the source has vanished is given by
\begin{equation}
    \phi(t,x)=\phi_0(t,\vec{x})+\phi_J(t,\vec{x})               =\int\frac{\dt ^3p}{(2\pi)^3}\frac{1}{\sqrt{2\omega_{\vec{p}}}}\left\{\big(a_{\vec{p}}+\frac{\imagu J(p)}{\sqrt{2\omega_{\vec{p}}}}\big)e^{-\imagu(\omega_{\vec{p}}t-\vec{p}\vec{x})}+\big(a_{\vec{p}}^\dagger-\frac{\imagu J(-p)}{\sqrt{2\omega_{\vec{p}}}}\big)e^{\imagu(\omega_{\vec{p}}t-\vec{p}\vec{x})}\right\}
\end{equation}
This is described by effectively replacing annihilation and creation operators via
\begin{equation}
    a_{\vec{p}}\mapsto a_{\vec{p}}+\frac{\imagu J(p)}{\sqrt{2\omega_{\vec{p}}}}\,,\qquad a_{\vec{p}}^\dagger\mapsto a_{\vec{p}}^\dagger-\frac{\imagu J(-p)}{\sqrt{2\omega_{\vec{p}}}}
\end{equation}
These replacements are of course compatible, considering that for a real source $J(p)=J^*(-p)$. Since $J(p)$ is just a $\mathbb{C}$-number, it does not alter the commutation relations from which the Hamiltonian and number operator are derived. The number density after the source has vanished, starting from the initial vacuum state $\ket{0}$ without any particles, is given by
\begin{equation}
    n_{\vec{p},J}=\bra{0}\Big(a_{\vec{p}}+\frac{\imagu J(p)}{\sqrt{2\omega_{\vec{p}}}}\Big)\Big(a_{\vec{p}}^\dagger-\frac{\imagu J^*(p)}{\sqrt{2\omega_{\vec{p}}}}\Big)\ket{0}=\frac{1}{2\omega_{\vec{p}}}\abs{J(p)}^2
\end{equation}

\subsection{Complex Scalar Field}

The derivation for the free scalar field is completely analogous. The identity $J(p)=J^*(-p)$ may not be used anymore. Instead, $J(p)$ contributes to the spectrum of particles, whereas $J(-p)$ contributes to the spectrum of antiparticles, in the following way:
\begin{equation}
    a_{\vec{p}}\mapsto a_{\vec{p}}+\frac{\imagu J(p)}{\sqrt{2\omega_{\vec{p}}}}\,,\qquad b_{\vec{p}}^\dagger\mapsto b_{\vec{p}}^\dagger-\frac{\imagu J(-p)}{\sqrt{2\omega_{\vec{p}}}}
\end{equation}
The particle and antiparticle momentum space number densities, or spectra, induces by the source $J$ are now
\begin{equation}
    n_{\vec{p},J}=\frac{1}{2\omega_{\vec{p}}}\abs{J(p)}^2\,,\qquad \overline{n}_{\vec{p},J}=\frac{1}{2\omega_{\vec{p}}}\abs{J(-p)}^2
\end{equation}

\subsection{Extracting the Source from the Late Time Field}

Equation \eqref{eq:ClassicalFieldAfterSource} allows use to extract
\begin{subequations}
    \begin{align}        
        J(p)e^{-\imagu\omega_{\vec{p}}t}&=\int\dt^3x\Big(-\imagu\omega_{\vec{p}}\phi_J(x)+\big(\partial_t\phi_J(x)\big)\Big)e^{-\imagu\vec{p}\vec{x}}\\
        J(p)&=\int\dt^3x\Big(\phi_J(x)\overset{\leftarrow}{\partial_t}e^{\imagu(\omega_{\vec{p}}t-\vec{p}\vec{x})}-\phi_J(x)\overset{\rightarrow}{\partial_t}e^{\imagu(\omega_{\vec{p}}t-\vec{p}\vec{x})}\Big)\\
        J(-p)e^{\imagu\omega_{\vec{p}}t}&=-\int\dt^3x\Big(-\imagu\omega_{\vec{p}}\phi_J(x)-\big(\partial_t\phi_J(x)\big)\Big)e^{\imagu\vec{p}\vec{x}}\\
        J(-p)&=\int\dt^3x\Big(\phi_J(x)\overset{\leftarrow}{\partial_t}e^{-\imagu(\omega_{\vec{p}}t-\vec{p}\vec{x})}-\phi_J(x)\overset{\rightarrow}{\partial_t}e^{-\imagu(\omega_{\vec{p}}t-\vec{p}\vec{x})}\Big)
    \end{align}
    \label{eq:SourceFieldRelation}
\end{subequations} 