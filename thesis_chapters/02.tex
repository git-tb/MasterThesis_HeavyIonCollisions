\chapter{Fluid Picture in Heavy Ion Collisions}

\section{Fundamentals of Relativistic Fluid Dynamics}

If many quantum mechanical particles are involved in the description of a process, finding the exact unitary time evolution of the system state is practically unfeasible. On one hand, explicit analytic expressions for example for eigenstates of the underlying Hamiltonian are rarely accessible, on the other hand numerical algorithms might only scale poorly with the system size, depending on symmetry assumptions etc.

The fundamental idea of thermodynamics is to describe systems of many interacting particles with a set of only a few possibly spacetime dependent variables, such as temperature $T(x)$, particle number $N$, system energy $E$ and volume $V$, among others. To do so, a small number of assumptions and fundamental laws are imposed in order to derive statements about observables of the macroscopic systems and how they arise from a statistical mixture of all possible microscopic states the system could occupy. A crucial part of this is the assumption of thermodynamic equilibrium, meaning a systems extensive parameters do not change. \todo{Work out meaning of equilibrium and out of equilibrium processes}

Based on the ideas of thermodynamics, the field of fluid dynamics tries to predict the collective dynamics of a system of interacting particles.
Dynamics are generated by equations of motion involving the energy-momentum tensor $T^{\mu\nu}(x)$ \cite{Ollitrault_2008},
\begin{equation}
    \begin{split}
        T^{00}\dots\,&\text{energy density,}\\
        T^{0j}\dots\,&\text{$j$th momentum component,}\\
        T^{i0}\dots\,&\text{energy flux in direction $i$,}\\
        T^{ij}\dots\,&\text{flux of $j$th momentum component in direction $i$.}\\
    \end{split}
\end{equation}
It is the conserved current associated to the conservation of 4-momentum, i.e. the symmetry under spacetime translations. \todo{Why does this coincide with the definition $T^{\mu\nu}$ from a Lagrangian?} In a local rest frame the energy-momentum tensor should have the form of static equilibrium \todo{Be specific}. There should be further no flow of particle number or entropy, specifying the form of the particle and entropy current $N^\mu$ and $S^\mu$. In a rest frame one finds
\begin{equation}
    T^{\mu\nu}_{RF}  =
        \begin{pmatrix}
            \epsilon & 0 & 0 & 0 \\
            0        & p & 0 & 0 \\
            0        & 0 & p & 0 \\
            0        & 0 & 0 & p
        \end{pmatrix}\,,\qquad
        N^\mu_{RF}       =
        \begin{pmatrix}
            n\\0\\0\\0
        \end{pmatrix}\,,\qquad
        S^\mu_{RF}       =
        \begin{pmatrix}
            s\\0\\0\\0
        \end{pmatrix}
        \label{eq:FluidMechanics_RestFrame_Quantities}
\end{equation}
with energy density $\epsilon$ and pressure $p$. \todo{Check why kinetic and thermodynamic pressure coincide.} In a general frame these quantitites are obtained by applying a Lorentz boost to \eqref{eq:FluidMechanics_RestFrame_Quantities} and read \cite{Rischke_2022,Weinberg_2008}
\begin{equation}
        T^{\mu\nu}=(\epsilon+p)u^\mu u^\nu+pg^{\mu\nu}\,,\qquad
        N^\mu       =nu^\mu\,,\qquad
        S^\mu       =su^\mu\,,
    \end{equation}
where $u^\mu=(\gamma,\gamma\mathbf{v})$ is the local fluid 4-velocity. Describing wordlines of massive particles, it is a timelike vector, normalized to ${u_\mu u^\mu=-1}$. The term proportional to pressure appearing in the energy momentum tensor is the projector ${\Delta^{\mu\nu}=g^{\mu\nu}+u^\mu u^\nu}$ orthogonal to $u^\mu$. It fulfills
\begin{equation}
    \Delta^{\mu\nu}=\Delta^{\nu\mu}\,,\qquad u_\mu\Delta^{\mu\nu}=0\,,\qquad\Delta^\mu_\lambda\Delta^{\lambda\nu}=\Delta^{\mu\nu}\,\qquad\Delta^\mu_\mu=3=d-1\,.
    \label{eq:FluidMechanics_ProjProperties}
\end{equation}

Local energy-momentum conservation and particle number conservation are encoded by the continuity equations
    \begin{equation}
        \partial_\mu T^{\mu\nu}  =0\,,\qquad
        \partial_\mu N^\mu       =0\,,
    \end{equation}
which constitutes 5 scalar equations for 6 unkwon functions: $\epsilon(x), P(x), n(x), \mathbf{v}(x)$. The equation of state $p=p(\epsilon,n)$ closes the system and captures the characteristics of the matter in question.

\todo{VISCOUS CORRECTIONS}


%====================================================================
\section{Bjorken Model}









\section{Converting Spectra between Coordinate Systems}
\label{sec:SpectraCoordinateSystem}

Consider the coordinate change in momentum space
\begin{equation}
    \left\{
    \begin{split}
        p^x&=p_T\cos\varphi_p\\
        p^y&=p_T\sin\varphi_p\\
        p^z&=m_T\sinh\eta_p\\
        p^t&=m_T\cosh\eta_p
    \end{split}
    \right.\qquad\iff\qquad
    \left\{
    \begin{split}
        p_T&=\sqrt{(p^x)^2+(p^y)^2}\\
        \varphi_p&=\arctan(p^y/p^x)\\
        m_T&=\sqrt{(p^t)^2-(p^z)^2}\\
        \eta_p&=\artanh(p^z/p^t)
    \end{split}
    \right.
\end{equation}
with Jacobian
\begin{equation}
    \big\vert\frac{\partial(p_T,\varphi_p,m_T,\eta_p)}{\partial(p^x,p^y,p^z,p^t)}\big\vert=\frac{1}{m_T p_T}
\end{equation}
Let $f(p^\mu)$ be some distribution function and $F$ its momentum space integral evaluated on the momentum shell and future directed momenta.

\begin{subequations}
    \begin{align}
        F=\int\frac{\dt^4p_{\text{cart}}}{(2\pi)^4}\delta(p^2+m^2)\Theta(p^t)f(p^\mu) & =\int\frac{\dt p^t}{2\pi}\int\frac{\dt^3p_{\text{cart}}}{(2\pi)^3}\frac{1}{2\omega_{\vec{p}}}\big(\delta(p^t-\omega_{\vec{p}})+\delta(p^t+\omega_{\vec{p}})\big)\Theta(p^t)f(p^\mu) \\
                                                                                      & =\frac{1}{2\pi}\int\frac{\dt^3p_{\text{cart}}}{(2\pi)^3}\frac{1}{2\omega_{\vec{p}}}f(p^\mu)\big\vert_{p^t=\omega_{\vec{p}}}
    \end{align}
    On the other hand
    \begin{align}
        F=\int\frac{\dt^4p_{\text{cart}}}{(2\pi)^4}\delta(p^2+m^2)\Theta(p^t)f(p^\mu) & =\frac{1}{(2\pi)^4}\int_0^\infty \dt p_T\int_0^\infty \dt m_T\int_{-\infty}^\infty \dt\eta_p\int_0^{2\pi}\dt\varphi_pm_T p_T\times\nonumber \\
                                                                                      & \phantom{=}\qquad\times\delta((p_T)^2-(m_T)^2+m^2)f(p^\mu)                                                                                              \\
        \intertext{assume $f(p^\mu)=f(p_T,m_T,\eta_p)$ and perform the $\varphi$-integraion as well as $m_T$-integration, where $\omega^{\perp,2}=m^2+p^{\perp,2}$ is defined}
                                                                                      & =\frac{1}{(2\pi)^3}\int_0^\infty\dt p_T\int_{-\infty}^\infty\dt\eta_p\frac{p_T}{2}f(p^\mu)\big\vert_{m_T=\omega_T}
    \end{align}
    leding to the important result
    \begin{equation}
        \frac{1}{2\pi}\int\frac{\dt^3p_{\text{cart}}}{(2\pi)^3}\frac{1}{2\omega_{\vec{p}}}f(p^\mu)\big\vert_{p^t=\omega_{\vec{p}}}=\frac{1}{(2\pi)^3}\int_0^\infty\dt p_T\int_{-\infty}^\infty\dt\eta_p\frac{p_T}{2}f(p^\mu)\big\vert_{m_T=\omega_T}
    \end{equation}
\end{subequations}

Since the restrictions $p^t=\omega_{\vec{p}}$ and $m_T=\omega_T$ are equivalent (considering the parametrization that already satisfies $p^t=p_T\cosh\eta_p\geq 0$) we find
\begin{equation}
    \omega_{\vec{p}}\frac{\dt F}{\dt p^x\dt p^y\dt p^z}=\frac{1}{2\pi p_T}\frac{\dt F}{\dt p_T\dt\eta_p}
\end{equation}
The result applies to the case
\begin{equation}
    f(p^\mu)\big\vert_{p^t=\omega_{\vec{p}}}=2\omega_{\vec{p}}\cdot 2\pi\cdot n(\vec{p})
\end{equation}
and $F=N$. \textbf{\textcolor{red}{This holds for the symmetric range $\eta\in(-\infty,\infty)$. Restricting to $\eta\in[0,\infty)$ we find
\begin{equation}
    2\omega_{\vec{p}}\frac{\dt F}{\dt p^x\dt p^y\dt p^z}=\frac{1}{2\pi p_T}\frac{\dt F}{\dt p_T\dt\eta_p}
\end{equation}
}}