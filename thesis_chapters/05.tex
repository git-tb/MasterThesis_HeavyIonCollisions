\chapter{Condensate Spectrum on the Freezeout Surface}

Putting together equations \eqref{eq:SpectraConversion}, \eqref{eq:NumberDensity}, \eqref{eq:SpectraFromSource} and \eqref{eq:SourceFieldRelation} captures the essential procedure we want to pursue to find the particle spectra. We are left to come up with reasonable initial conditions for the fields of interest after the interaction in the fireball of QGP has deceased. The strong interactions are found to be described fairly accurately simulations of relativistic hydrodynamics. Therefore we attempt to define initial data at the point where the hydro simulation is stopped and the particle enter the non-interacting or free streaming phase. This point is called the Freezeout and is not described by a flat $\mathbb{R}^3$ hypersurface of constant lab time $t$, but rather by a numerically determined curve in the $\tau$-$r$-plane in Bjorken coordinates. We thus need to adapt our calculations for that case. 

\section{Invariance of Fourier Transform w.r.t. Deformations of the Hypersurface}
\label{subsec:FourierDeformHypersurface}

Let $\phi_1,\phi_2$ be fields of equal mass evolving according to the KG equation. Then the current
\begin{equation}
    J_\mu[\phi_1,\phi_2]=-\imagu(\phi_1\partial_\mu\phi_2^*-(\partial_\mu\phi_1)\phi_2^*)\eqdef-\imagu\phi_1\overset{\leftrightarrow}{\partial_\mu}\phi_2^*
\end{equation}
with the antisymmetrized two-sided derivative $\overset{\leftrightarrow}{\partial_\mu}=\overset{\rightarrow}{\partial_\mu}-\overset{\leftarrow}{\partial_\mu}$ is conserved. Recall Gauß law on Cauchy hypersurfaces (up to a sign depending on the metric signature)
\begin{equation}
    \int_\Omega\dt \Omega\,\nabla^\mu J_\mu=\int_{\partial\Omega}\dt\sigma^\mu\,J_\mu
\end{equation}
with $\dt\sigma_\mu$ the outwards oriented surface normal of the spacetime volume $\Omega$. The bilinear form
\begin{equation}
    (\phi_1,\phi_2)_\Sigma=\int_\Sigma\dt\Sigma^\mu\,J_\mu[\phi_1,\phi_2]=-\imagu\int_\Sigma\dt\Sigma^\mu\,\phi_1\overset{\leftrightarrow}{\partial_\mu}\phi_2^*
    \label{eq:InvariantInnterProduct}
\end{equation}
is therefore independent of the choice of (Cauchy) hypersurface $\Sigma$ (if $\partial\Sigma$ is changed, one must carefully check for further contributions in Gauß law). Let 
\begin{equation}
    u_{\vec{p}}(t,\vec{x})=\exp(-\imagu(\omega_{\vec{p}}t-\vec{p}\vec{x}))\,,\qquad u_{\vec{p}}^*(t,\vec{x})=\exp(\imagu(\omega_{\vec{p}}t-\vec{p}\vec{x}))
\end{equation}
be the positive and negative frequency eigensolutions to the free Klein-Gordon equation. They form an orthogonal system with respect to the inner product defined above,
\begin{gather}
    (u_{\vec{p}},u_{\vec{q}})_{\Sigma_t}=(2\omega_{\vec{p}})(2\pi)^3\delta^{(3)}(\vec{p}-\vec{q})\,,\qquad (u_{\vec{p}}^*,u_{\vec{q}}^*)_{\Sigma_t}=-(2\omega_{\vec{p}})(2\pi)^3\delta^{(3)}(\vec{p}-\vec{q})\,,\\
    (u_{\vec{p}},u_{\vec{q}}^*)_{\Sigma_t}=0
\end{gather}
with the relations stated here on a a hypersurface $\Sigma_t$ where $t=\text{const.}$. This means that the Fourier coefficients, or equivalently annihilation and creation operators after quantization, for example in equation \eqref{eq:CanonicalQuant_RealScalar}, can be extraced via
\begin{equation}
    \sqrt{2\omega_{\vec{p}}}a_{\vec{p}}=(\phi,u_{\vec{p}})_{\Sigma_t}\,,\qquad \sqrt{2\omega_{\vec{p}}}a_{\vec{p}}^\dagger=-(\phi,u_{\vec{p}}^*)_{\Sigma_t}
\end{equation}
This leads of course to the same statement as in equations \eqref{eq:AnnCrePhiPi_Relation}.

The equations in \eqref{eq:SourceFieldRelation} are also precisely of this form, namely
\begin{subequations}
    \begin{align}
        J(p)&=-\int_{\Sigma_t}\dt\Sigma^\mu\,\phi_J\overset{\leftrightarrow}{\partial_\mu}u_{\vec{p}}^*=-\imagu(\phi_J,u_{\vec{p}})_{\Sigma_t}\\
        J(-p)&=-\int_{\Sigma_t}\dt\Sigma^\mu\,\phi_J\overset{\leftrightarrow}{\partial_\mu}u_{\vec{p}}=-\imagu(\phi_J,u_{\vec{p}}^*)_{\Sigma_t}
    \end{align}
    \label{eq:SourceFieldRelation_InnerProduct}
\end{subequations}
and are thus independent of the choice of Cauchy hypersurface.\\
\debugbox{
    \begin{minipage}{\linewidth}
        \centering
        \includegraphics[width=0.4\linewidth]{images/FreezeOutSurface.pdf}
        \captionof{figure}{Freezeout surface in $\tau$-$r$-plane \cite{KirchnerEtAl_2023}.}
        \label{fig:FreezeOutSurface_rtau}
    \end{minipage}
}
Consider the freezeout on the hypersurface depicted in figure \ref{fig:FreezeOutSurface_rtau}. Following the reasoning from \cite{KirchnerEtAl_2023}, we wish to apply Gauß law, taking care of the different signature compared to the usual version of Gauß law in Euclidean space. A calculation highlighting the difference between Euclidean and Minkowski signature is performed in Appendix \ref{sec:Apdx_GaußLawMinkowski}.

Assuming that the condensate has no contributions at large rapidities ${\eta\to\pm\infty}$, one can deform the hypersurface $\Sigma_t$ at large lab time ${t=\const}$ into a hypersurface of large Bjorken time $\Sigma_{\tau\gg\tau_L}$ at a $\tau$ much larger then the lifetime $\tau_L$ of the fireball, using Gauß law applied to the spacetime volume enclosed by these two Cauchy surfaces. There is no contribution from the $\tau$-axis, because the set of points where $r=0$ has vanishing 3-volume. Assuming further that the condensate contribution as a function in phase space $f_{\text{cond}}(x^\mu,\vec{p})$ vanishes on $\Sigma_{\rom{2}}$ and $\Sigma_{\rom{5}}$, i.e. is contained within the union of all light cones starting on the freezeout surface ${\Sigma_{\text{fo}}\equiv\Sigma_{\rom{1}}\cap\Sigma_{\rom{3}}}$, the contributions on $\Sigma_{\rom{4}}$ and $\Sigma_{\rom{1}}$ coincide. From intuition about the Euclidean version of Gauß law one would expect a change in sign, given the normal vectors on these surface oriented as in figure \ref{fig:FreezeOutSurface_rtau}. However, since $\Sigma_{\rom{1}}$ is a timelike hypersurface there is another sign-flip compared to the spacelike hypersurface $\Sigma_{\rom{4}}$.

To summarize, the hypersurface on which the inner products are computed can be changed according to
\begin{equation}
    (\phi_J,u_{\vec{p}}^{(*)})_{\Sigma_t}=(\phi_J,u_{\vec{p}}^{(*)})_{\Sigma_{\tau\gg\tau_L}}=(\phi_J,u_{\vec{p}}^{(*)})_{\Sigma_{\rom{3}}\cup\Sigma_{\rom{4}}}=(\phi_J,u_{\vec{p}}^{(*)})_{\Sigma_{\rom{3}}\cup\Sigma_{\rom{1}}}\equiv(\phi_J,u_{\vec{p}}^{(*)})_{\Sigma_{\text{fo}}}\,.
    \label{eq:InnerproductHypersurfaceDeformation}
\end{equation}

\section{Metric on the Freezeout Surface}

To evaluate the inner product \eqref{eq:InvariantInnterProduct} explicitly on the freezeout surface, we need to find the induced metric and surface element of the hypersurface as a submanifold of Minkowski space $\mathbb{R}^{(1,3)}$ using a standard procedure.

The freezeout hypersurface is parametrized as $\Sigma_{\text{fo}}=\{x^\mu\in\mathbb{R}^{(1,3)}\vert (\tau,r)=(\tau(\alpha),r(\alpha))\}$ with $\tau,r$ defined by the coordinate transformation \eqref{eq:BjorkenCoords_PositionSpace}. We will need the oriented surface normal $\dt\Sigma^\mu$ on this hypersurface. Recall the metric ${g_{\mu\nu}=\text{diag}(-1,1,\tau^2,r^2)}$ in Bjorken coordinates $(\tau,r,\eta,\varphi)$. Orthonormal tangent vectors on the freezeout hypersurface are ${(\hat\partial_\varphi)^\mu=(0,0,0,r^{-1})=r^{-1}(\partial_\varphi)^\mu}$, ${(\hat\partial_\eta)^\mu=(0,0,\tau^{-1},0)=\tau^{-1}(\partial_\eta)^\mu}$ and ${(\hat\partial_\alpha)^\mu=\sqrt{\vert r^{\prime 2}(\alpha)-\tau^{\prime 2}(\alpha)\vert}^{-1}(\tau^{\prime}(\alpha),r^{\prime}(\alpha),0,0)=D(\alpha)(\partial_\alpha)^\mu}$ with ${D(\alpha)=\sqrt{\vert r^{\prime 2}(\alpha)-\tau^{\prime 2}(\alpha)\vert}^{-1}}$. The projector on the hypersurface is
    \begin{multline}
        \gamma_{\mu\nu}=(\hat\partial_\varphi)_\mu(\hat\partial_\varphi)_\nu+(\hat\partial_\eta)_\mu(\hat\partial_\eta)_\nu+(\hat\partial_\alpha)_\mu(\hat\partial_\alpha)_\nu\\
        =\begin{pmatrix}
            D^2(\alpha)\tau^{\prime2}(\alpha)               & -D^2(\alpha)\tau^\prime(\alpha)r^\prime(\alpha) & 0      & 0   \\
            -D^2(\alpha)\tau^\prime(\alpha)r^\prime(\alpha) & D^2(\alpha)r^{\prime2}(\alpha)                  & 0      & 0   \\
            0                                               & 0                                               & \tau^2 & 0   \\
            0                                               & 0                                               & 0      & r^2
        \end{pmatrix}
    \end{multline}
The normal of the hypersurface, defined by its orthogonality to all tangent vectors, is ${n^\mu\equiv(\hat\partial_\alpha^\perp)^\mu=D(\alpha)(r^\prime(\alpha),\tau^\prime(\alpha),0,0)}$ and is timelike (spacelike) when ${r^\prime>\tau^\prime}$ (${r^\prime<\tau^\prime}$). Naturally ${\gamma_{\mu\nu}n^\nu=0}$. Using that
    \begin{equation}
        (\partial_\alpha)^\nu\gamma_{\mu\nu}(\partial_\alpha)^\mu=\begin{pmatrix}
            \tau^\prime \\r^\prime
        \end{pmatrix}^T\begin{pmatrix}
            -\tau^\prime \\
            r^\prime
        \end{pmatrix}=D^{-2}\,,
    \end{equation}
the induced hypersurface metric in coordinates ${x^i=(\alpha,\eta,\varphi)}$ reads
    \begin{equation}
        \gamma_{ij}=\text{diag}(D^{-2}(\alpha),\tau^2(\alpha),r^2(\alpha))
    \end{equation}
and the volume element is given by ${\dt\Sigma=\det(\gamma_{ij})\dt\alpha\dt\eta\dt\varphi=r(\alpha)\tau(\alpha) D^{-1}(\alpha)\dt\alpha\dt\eta\dt\varphi}$. The oriented surface element is 
\begin{equation}
        \dt\Sigma^\mu=n^\mu\dt\Sigma=r(\alpha)\tau(\alpha)(r^\prime(\alpha),\tau^\prime(\alpha),0,0)\dt\alpha\dt\eta\dt\varphi\,,
        \label{eq:FreezeoutSurface_SurfaceElementOriented}
\end{equation}
which is the main result from this section.

\section{Computing the Inner Product}

Coming back to the opening paragraph of this chapter, we now have all the building blocks to derive an explicit formula ready for numerical evaluation on the freezeout surface. The main equations to be linked together are now \eqref{eq:SpectraConversion}, \eqref{eq:NumberDensity}, \eqref{eq:SpectraFromSource} and \eqref{eq:SourceFieldRelation_InnerProduct} together with \eqref{eq:InnerproductHypersurfaceDeformation} and \eqref{eq:FreezeoutSurface_SurfaceElementOriented}.

The projection of the derivative onto the surface normal of $\Sigma_{\text{fo}}$  is (omitting the $\alpha$-dependence) 
\begin{equation}
    \dt\Sigma^\mu\partial_\mu=(r^\prime\partial_\tau+\tau^\prime\partial_r)\cdot r\tau\,\dt\alpha\dt\eta\dt\varphi\,.
\end{equation}
and there are a priori 3 integrals to be evaluated numerically. Making use of boost and rotational symmetry, one can perform the $\eta$- and $\varphi$-integration analytically. To do so, one can apply the following integrals related to Bessel functions: \url{https://dlmf.nist.gov/10.9}
\begin{subequations}
    % \begin{align}
    %     \int_0^{2\pi}\dt\varphi e^{\pm\imagu a\cos\varphi}&=\int_0^{2\pi}\big(\cos(a\cos\varphi)\pm\imagu\sin(a\cos\varphi)\big)=2\int_0^\pi\cos(a\cos\varphi)\\
    %     &=2\pi J_0(a)\\
    %     \int_{-\infty}^\infty\dt\eta e^{\pm\imagu a\cosh\eta}&=2\int_0^\infty\dt\eta\big(\cos(a\cosh\eta)\pm\imagu\sin(a\cosh\eta)\big)\\
    %     &=\pi\big(-Y_0(a)\pm\imagu J_0(a)\big)
    %     &=\pm\pi\imagu(J_0(a)\pm\imagu Y_0(a))=\begin{cases}
    %         +\pi\imagu H^{(1)}_0(a)&\text{for "+"}\\
    %         -\pi\imagu H^{(2)}_0(a)&\text{for "-"}
    %     \end{cases}
    % \end{align}
    \begin{equation}
        \int_0^{2\pi}\dt\varphi e^{\pm\imagu a\cos\varphi}=2\pi J_0(a)\,,\qquad
        \int_{-\infty}^\infty\dt\eta e^{\pm\imagu a\cosh\eta}=\pi\big(-Y_0(a)\pm\imagu J_0(a)\big)\,.
    \end{equation}
    Additionally to the integral representations, the following computation makes use of \url{https://dlmf.nist.gov/10.4}
    \begin{equation}
        J_{0}^\prime(x)=-J_1(x)\,,\qquad Y_0^\prime(x)=-Y_1(x)\,.
    \end{equation}
\end{subequations}
Straightforward substitution of the different formula into each other leads to the central result
\begin{multline}
    J(\pm p) =2\pi^2\int_0^\pi\dt\alpha\tau r\Bigg[(r^\prime\partial_\tau+\tau^\prime\partial_r)\phi(\tau,r)\Big[J_0(r p_T)\times\big(-Y_0(\tau\omega_T)\pm\imagu J_0(\tau\omega_T)\big)\Big]+\nonumber                                                                                             \\
                       + \phi(\tau,r)\Big[\tau^\prime\times p_T J_1(r p_T)\times\big(-Y_0(\tau\omega_T)\pm\imagu J_0(\tau\omega_T)\big)+\nonumber                                                                                                                                       \\
                       \qquad\phantom{+\phi(\tau,r)\Big[}+r^\prime\times J_0(r p_T)\times\omega_T\big(-Y_1(\tau\omega_T)\pm\imagu J_1(\tau\omega_T)\big)\Big]\Bigg]\,.
\end{multline}
This can be evaluated numerically, once the shape of the freezeout surface is provided in the form of 2 function $\tau(\alpha)$ and $r(\alpha)$, and the initial data ${\phi\vert_{\Sigma_{\text{fo}}}}$ and ${\dt\Sigma^\mu\partial_\mu\phi\vert_{\Sigma_{\text{fo}}}}$ is specified. The necessity of 2 scalar functions as initial data is consistent with the idea that a solution to the Klein-Gordon equation has 2 scalar function-valued degrees of freedom.

\begin{subequations}
    \begin{align}
        J(\pm p) & =-\int_{-\infty}^\infty\dt\eta\int_0^{2\pi}\dt\varphi\int_0^\pi\dt\alpha\tau r\Bigg[\phi(\tau,r)\big(r^\prime\overset{\leftrightarrow}{\partial_\tau}+\tau^\prime\overset{\leftrightarrow}{\partial_r}\big)e^{\pm\imagu(\tau \omega_T\cosh(\eta-\eta_p)-r p_T\cos(\varphi-\varphi_p))}\Bigg] \\
                           & =-\int_{-\infty}^\infty\dt\eta\int_0^{2\pi}\dt\varphi\int_0^\pi\dt\alpha\tau r\Bigg[\phi(\tau,r)\big(r^\prime\overset{\leftrightarrow}{\partial_\tau}+\tau^\prime\overset{\leftrightarrow}{\partial_r}\big)e^{\pm\imagu(\tau \omega_T\cosh\eta-r p_T\cos\varphi)}\Bigg]                      \\
                           & =-2\pi^2\int_0^\pi\dt\alpha\tau r\Bigg[\phi(\tau,r)(r^\prime\overset{\leftrightarrow}{\partial_\tau}+\tau^\prime\overset{\leftrightarrow}{\partial_r})\Big[J_0(r p_T)\times\big(-Y_0(\tau\omega_T)\pm\imagu J_0(\tau\omega_T)\big)\Big]\Bigg]                                                 \\
                           & =2\pi^2\int_0^\pi\dt\alpha\tau r\Bigg[(r^\prime\partial_\tau+\tau^\prime\partial_r)\phi(\tau,r)\Big[J_0(r p_T)\times\big(-Y_0(\tau\omega_T)\pm\imagu J_0(\tau\omega_T)\big)\Big]+\nonumber                                                                                             \\
                           & \phantom{=}\qquad + \phi(\tau,r)\Big[\tau^\prime\times p_T J_1(r p_T)\times\big(-Y_0(\tau\omega_T)\pm\imagu J_0(\tau\omega_T)\big)+\nonumber                                                                                                                                       \\
                           & \phantom{=}\qquad\phantom{+\phi(\tau,r)\Big[}+r^\prime\times J_0(r p_T)\times\omega_T\big(-Y_1(\tau\omega_T)\pm\imagu J_1(\tau\omega_T)\big)\Big]\Bigg]
    \end{align}
\end{subequations}

\begin{subequations}
    \begin{align}
        J(\overset{(-)}{+}p) & =-\int_{-\infty}^\infty\dt\eta\int_0^{2\pi}\dt\varphi\int_0^\pi\dt\alpha\tau r\Bigg[\phi(\tau,r)\big(r^\prime\overset{\leftrightarrow}{\partial_\tau}+\tau^\prime\overset{\leftrightarrow}{\partial_r}\big)e^{\overset{(-)}{+}\imagu(\tau \omega_T\cosh(\eta-\eta_p)-r p_T\cos(\varphi-\varphi_p))}\Bigg] \\
                           & =-\int_{-\infty}^\infty\dt\eta\int_0^{2\pi}\dt\varphi\int_0^\pi\dt\alpha\tau r\Bigg[\phi(\tau,r)\big(r^\prime\overset{\leftrightarrow}{\partial_\tau}+\tau^\prime\overset{\leftrightarrow}{\partial_r}\big)e^{\overset{(-)}{+}\imagu(\tau \omega_T\cosh\eta-r p_T\cos\varphi)}\Bigg]                      \\
                           & =\overset{(+)}{-}2\pi^2\imagu\int_0^\pi\dt\alpha\tau r\Bigg[\phi(\tau,r)(r^\prime\overset{\leftrightarrow}{\partial_\tau}+\tau^\prime\overset{\leftrightarrow}{\partial_r})\Big[J_0(r p_T)\times H_0^{\overset{(2)}{(1)}}(\tau\omega_T)\Big]\Bigg]                                                 \\
                           & =\overset{(-)}{+}2\pi^2\imagu\int_0^\pi\dt\alpha\tau r\Bigg[(r^\prime\partial_\tau+\tau^\prime\partial_r)\phi(\tau,r)\Big[J_0(r p_T)\times H_0^{\overset{(2)}{(1)}}(\tau\omega_T)\Big]+\nonumber                                                                                             \\
                           & \phantom{=}\qquad + \phi(\tau,r)\Big[\tau^\prime\times p_T J_1(r p_T)\times H_0^{\overset{(2)}{(1)}}(\tau\omega_T)+r^\prime\times J_0(r p_T)\times\omega_T H_1^{\overset{(2)}{(1)}}(\tau\omega_T)\Big]\Bigg]
    \end{align}
\end{subequations}

\section{Models for Initial Data}

If $J^\mu_{V,A}$ are assumed to be parallel to $u^\mu$, the assumption that all fields change only in direction of $u^\mu$ is consistent.

We shall try to investigate certain classes of initial conditions:
\begin{impt}[Possibilies to Fix Initial Data]{impt:InitialData}
\begin{enumerate}
    \item prescribe fields and orhtogonal derivatives directly \dots
    \begin{itemize}
        \item \dots in the sense of an expansion (starting with constants, Gauß modes?, \dots)
        \item \dots as function of $(\tau,r)$
    \end{itemize}
    \item prescribe other physically accessible parameters like energy density $\epsilon$, pressure $p$, fluid velocity $u^\mu$\dots
\end{enumerate}
\end{impt}

\subsection{Real Fields}
\label{sec:FluidFromRealScalar}

Let $\phi$ be a real scalar field, e.g. $\phi\in\{\sigma,\pi^0\}$. The only availabe $4$-vector in the fluid theory is $u_\mu$. It is thus intuitive to try to identify the real-valued $4$vector $\partial_\mu\phi\sim u_\mu$. Taking the normalization $u_\mu u^\mu=-1$ into account, one finds
\begin{equation}
    u_\mu=\frac{\partial_\mu\phi}{\chi}\,,\qquad0<\chi^2\defeq-(\partial_\mu\phi)(\partial^\mu\phi)
\end{equation}
This does not restrict the initial data enough to uniquely specify the resulting spectrum. We are left to prescribe two functions on the freezeout surface, namely $\phi\vert_{\Sigma_{\text{fo}}}$ and $\chi\vert_{\Sigma_{\text{fo}}}$.

Regarding the second option in \ref{impt:InitialData}, from the fluid theory, we could try to match the energy density of the hypothetical superfluid
\begin{subequations}
    \begin{align}
        \epsilon_{s,\phi}=u_\mu u_\nu T^{\mu\nu}_{\phi}&=\frac{(\partial_\nu\phi)(\partial_\mu\phi)}{\chi^2}\Big((\partial^\mu\phi)(\partial^\nu\phi)+g^{\mu\nu}\big(-\frac{1}{2}(\partial_\alpha\phi)(\partial^\alpha\phi)-\frac{1}{2}m_\phi^2(\phi)^2\big)\Big)   \\
        &=\chi^2-\big(\frac{1}{2}\chi^2-\frac{1}{2}m_\phi^2\phi^2\big)  \\
        &=\frac{1}{2}\big(m_\phi^2\phi^2+\chi^2\big)
    \end{align}
\end{subequations}
which changes along the freezeout surface as
    \begin{equation}
        \dt\epsilon=m_\phi^2\phi\dt\phi+\chi\dt\chi=m_\phi^2\phi(\partial_i\phi)\dt^i s+\chi\dt\chi=m_\phi^2\phi(\chi u_i)\dt^i s+\chi\dt\chi
    \end{equation}
where the sum over $i\in\{\tau,r\}$ is to be taken w.r.t. to a Euclidean metric, i.e.
\begin{equation}
    \dt\phi=\dt\tau\partial_\tau\phi+\dt r\partial_r\phi=\dt\alpha(\tau^\prime \underbrace{u_\tau}_{\textcolor{red}{=-u^\tau}}+r^\prime u_r)\chi\equiv\chi u_i\dt^is
\end{equation}
with $\dt^i s=(\partial x^\mu)/(\partial\alpha)\dt\alpha\equiv(\tau^\prime,r^\prime)\dt\alpha$ the displacement vector on the freezeout surface. For this prescription, the solution $\phi$ on the freezout surface thus needs to fulfill the ODE
\begin{equation}
    \dt\phi=\chi u_i\dt^i s\,,\quad
    \dt\chi=\frac{1}{\chi}\dt\epsilon-m_\phi^2\phi u_i\dt^i s
\end{equation}
The solutions $(\phi,\chi)$ of this ODE have 1 degree of freedom, e.g. the ratio of kinetic energy $\epsilon_{\text{kin}}=(1/2)\chi^2$ and $\epsilon_{\text{pot}}=(1/2)m_\phi^2(\phi)^2$ at $\alpha=0$. To be precise, choose $r\in[0,1]$ and set $\epsilon_{\text{pot}}\big\vert_{\alpha=0}=r\epsilon$ and $\epsilon_{\text{kin}}\big\vert_{\alpha=0}=(1-r)\epsilon$.

For the real scalar field $\pi^0$ one finds
\begin{equation}
    \dt\Sigma^\mu\partial_\mu=\chi(r^\prime \underbrace{u_\tau}_{\textcolor{red}{=-u^\tau}}+\tau^\prime u_r)\cdot r\tau\,\dt\alpha\dt\eta\dt\varphi\,.
\end{equation}

\paragraph{The Sign of $u^\mu\partial_\mu\phi$}

The 4-vector $u^\mu$ is timelike, ${u_\mu u^\mu=-1}$ and future oriented, $u^\tau>0$. Long after the collision, when the fireball expands and density decreases, the amplitude of the condensate can be expected to decrease in direction of the fluid flow, ${\dt(\phi)^2\sim\phi\dt\phi<0}$ where
\begin{equation}
    \dt\phi=(u^\tau\partial_\tau\phi+u^r\partial_r\phi)\dt\tau_E=u^\mu\partial_\mu\phi\dt\tau_E
\end{equation}
with the eigen time interval $\dt\tau_E>0$. Inserting ${\partial_\mu\phi=\chi u_\mu}$ gives
\begin{equation}
    \dt\phi=\chi u_\mu u^\mu\dt\tau_E=-\chi\dt\tau_E\,.
\end{equation}
In this case, $\phi\chi>0$ is the appropiate choice for a decreasing density of the condensate in the frame of the expanding fireball.

\subsection{Complex Fields}

Let $\phi$ be a complex field, e.g. $\phi\in\{\pi^+,\pi^-\}$ with $U(1)$ symmetry. In the case of $\pi^\pm$, this $U(1)$ symmetry is the symmetry under chiral rotations of $\pi^1\leftrightarrow\pi^2$. Using $\pi^1=(1/\sqrt{2})(\pi^++\pi^-)$ and $\pi^2=(\imagu/\sqrt{2})(\pi^+-\pi^-)$, as well as $\pi^\pm=\sqrt{n}\exp(\pm\imagu\theta)$, the conserved current is
\begin{align}
    \epsilon^{a12}J_V^{a,\mu}&=(\partial^\mu\pi^1)\pi^2-(\partial^\mu\pi^2)\pi^1\\
    &=\frac{\imagu}{2}\Big((\partial^\mu\pi^++\partial^\mu\pi^-)(\pi^+-\pi^-)-(\partial^\mu\pi^+-\partial^\mu\pi^-)(\pi^++\pi^-)\Big)\\
    &=\imagu\Big(-(\partial^\mu\pi^+)\pi^-+(\partial^\mu\pi^-)\pi^+\Big)\\
    &=\imagu\sqrt{n}\Big(-\big((\partial^\mu\sqrt{n})+\imagu\sqrt{n}(\partial^\mu\theta)\big)+\big((\partial^\mu\sqrt{n})-\imagu\sqrt{n}(\partial^\mu\theta)\big)\Big)\\
    &=n(\partial^\mu\theta)
\end{align}

The ansatz $J_V^{\mu,a}\sim u^\mu$ now implies
\begin{equation}
    u^\mu=\frac{\partial^\mu\theta}{\chi_\theta}\,,\qquad 0<\chi_\theta^2\defeq-(\partial_\mu\theta)(\partial^\mu\theta)
\end{equation}
Further imposing $(\partial^\mu\pi^a)\sim u^\mu$ leads to 
\begin{equation}
    u_\mu=\frac{\partial_\mu\sqrt{n}}{\chi_n}\,,\qquad 0<\chi_n^2\defeq-(\partial_\mu\sqrt{n})(\partial^\mu\sqrt{n})
\end{equation}
and a resulting energy density of
\begin{subequations}
    \begin{align}
        \epsilon_{s,\phi}=u_\mu u_\nu T^{\mu\nu}_{\phi}&=\frac{(\partial_\mu\theta)(\partial_\nu\theta)}{\chi_\theta^2}\Big(2(\partial^\mu\phi)(\partial^\nu\phi^*)+g^{\mu\nu}\big(-(\partial_\alpha\phi)(\partial^\alpha\phi^*)-m_\phi^2\phi\phi^*\big)\Big)\\
        &=n\chi_\theta^2+\chi^2_n+m_\phi^2n
    \end{align}
\end{subequations}
where the intermediate calculation
\begin{subequations}
    \begin{align}
        (\partial^\mu\phi)(\partial^\nu\phi^*)&=\big((\partial^\mu\sqrt{n})+\imagu\sqrt{n}(\partial^\mu\theta)\big)\big((\partial^\nu\sqrt{n})-\imagu\sqrt{n}(\partial^\nu\theta)\big)\\
        &=(\partial^\mu\sqrt{n})(\partial^\nu\sqrt{n})+n(\partial^\mu\theta)(\partial^\nu\theta)+\imagu\big(\sqrt{n}(\partial^\mu\theta)(\partial^\nu\sqrt{n})-\sqrt{n}(\partial^\nu\theta)(\partial^\mu\sqrt{n})\big)\\
        (\partial_\mu\phi)(\partial^\mu\phi^*)&=-\chi_n^2-n\chi_\theta^2
    \end{align}
\end{subequations}
is useful. Note how the imaginary part of this tensor is antisymmetric and thus does not contribute upon contraction with a symmetric tensor.

Considering the fact that we have already proposed ways to initialize the real valued fields $\pi^1$, $\pi^2$, we could also just simply construct $\pi^\pm=(1/\sqrt{2})(\pi^1\mp\imagu\pi^2)$ from these fields. Given $\partial^\mu\pi^{1,2}$, the density of the charged pion fields changes as
\begin{equation}
    \partial^\mu n=\partial^\mu(\pi^+\pi^-)=\frac{1}{2}\partial^\mu\big((\pi^1)^2+(\pi^2)^2\big)=(\partial^\mu\pi^1)\pi^1+(\partial^\mu\pi^2)\pi^2
\end{equation}

\section{Examplary Spectra}

This section is aimed toward finding some intuition for the relation between particle mass, initial conditions and resulting spectra.

First, investigate the spectrum of a real scalar field for constant values of the field an normal derivative.\\
\debugbox{
    \begin{minipage}{\linewidth}
        \centering
        \debugbox{
            \begin{minipage}{0.45\linewidth}
                \centering
                \includegraphics[width=\linewidth]{code/C++/DCCspec/data/images/spectra_real_constfield_20240822_161734_init.png}        
            \end{minipage}
        }
        \debugbox{
            \begin{minipage}{0.45\linewidth}
                \centering
                \includegraphics[width=\linewidth]{code/C++/DCCspec/data/images/spectra_real_constfield_20240822_161734_spec.png}        
            \end{minipage}
        }
    \end{minipage}
}
Spectra from field configurations of high ratio ${\Big\vert\frac{\text{amplitude}}{\text{derivative}}\Big\vert}$ tend to be more "bumpy"/oscillate quicker in momentum space.

These configurations are well suited to test separately the impact of the mass parameter in the spectrum computation, keeping the initial conditions fixed.\\
\debugbox{
    \begin{minipage}{\linewidth}
        \centering
        \debugbox{
            \begin{minipage}{0.45\linewidth}
                \centering
                \includegraphics[width=\linewidth]{code/C++/DCCspec/data/images/spectra_real_constfield_20240822_161734_masses_init.png}        
            \end{minipage}
        }
        \debugbox{
            \begin{minipage}{0.45\linewidth}
                \centering
                \includegraphics[width=\linewidth]{code/C++/DCCspec/data/images/spectra_real_constfield_20240822_161734_masses_spec.png}        
            \end{minipage}
        }
    \end{minipage}
}
The displayed graphs correspond to particle masses from 0.14 GeV (blue) to 0.8 GeV (red). Together with the scale on which the freezeout surface varies, the particle mass sets the main scale/width of the spectrum.


Next, test the $\epsilon=\const$ prescription of defining initial data, which yields a 1-parameter family of initial field configuration.\\
\debugbox{
    \begin{minipage}{\linewidth}
        \centering
        \debugbox{
            \begin{minipage}{0.45\linewidth}
                \centering
                \includegraphics[width=\linewidth]{code/C++/DCCspec/data/images/spectra_real_consteps_20240822_135426_init.png}        
            \end{minipage}
        }
        \debugbox{
            \begin{minipage}{0.45\linewidth}
                \centering
                \includegraphics[width=\linewidth]{code/C++/DCCspec/data/images/spectra_real_consteps_20240822_135426_spec.png}        
            \end{minipage}
        }
    \end{minipage}
}
The red graphs represent field configurations that are approximately constant on the ${\tau\approx\const}$ section of the freezeout surface. They seem to generate a distinct "break" in the spectrum, where the average slope (in the log scale) changes abruptly.

Let us also change the mass in this scenario. Now, the mass also influences the initial conditions and lead to more oscillations along the freezeout surface. The following graphs range from 0.14 GeV to 1 GeV particle mass.\\
\debugbox{
    \begin{minipage}{\linewidth}
        \centering
        \debugbox{
            \begin{minipage}{0.45\linewidth}
                \centering
                \includegraphics[width=\linewidth]{code/C++/DCCspec/data/images/spectra_real_consteps_20240826_143119_m226_init.png}        
            \end{minipage}
        }
        \debugbox{
            \begin{minipage}{0.45\linewidth}
                \centering
                \includegraphics[width=\linewidth]{code/C++/DCCspec/data/images/spectra_real_consteps_20240826_143119_m226_spec.png}        
            \end{minipage}
        }
    \end{minipage}
}

\debugbox{
    \begin{minipage}{\linewidth}
        \centering
        \debugbox{
            \begin{minipage}{0.45\linewidth}
                \centering
                \includegraphics[width=\linewidth]{code/C++/DCCspec/data/images/spectra_real_consteps_20240826_143151_m398_init.png}        
            \end{minipage}
        }
        \debugbox{
            \begin{minipage}{0.45\linewidth}
                \centering
                \includegraphics[width=\linewidth]{code/C++/DCCspec/data/images/spectra_real_consteps_20240826_143151_m398_spec.png}        
            \end{minipage}
        }
    \end{minipage}
}

\debugbox{
    \begin{minipage}{\linewidth}
        \centering
        \debugbox{
            \begin{minipage}{0.45\linewidth}
                \centering
                \includegraphics[width=\linewidth]{code/C++/DCCspec/data/images/spectra_real_consteps_20240826_143224_m570_init.png}        
            \end{minipage}
        }
        \debugbox{
            \begin{minipage}{0.45\linewidth}
                \centering
                \includegraphics[width=\linewidth]{code/C++/DCCspec/data/images/spectra_real_consteps_20240826_143224_m570_spec.png}        
            \end{minipage}
        }
    \end{minipage}
}

\debugbox{
    \begin{minipage}{\linewidth}
        \centering
        \debugbox{
            \begin{minipage}{0.45\linewidth}
                \centering
                \includegraphics[width=\linewidth]{code/C++/DCCspec/data/images/spectra_real_consteps_20240826_143253_m742_init.png}        
            \end{minipage}
        }
        \debugbox{
            \begin{minipage}{0.45\linewidth}
                \centering
                \includegraphics[width=\linewidth]{code/C++/DCCspec/data/images/spectra_real_consteps_20240826_143253_m742_spec.png}        
            \end{minipage}
        }
    \end{minipage}
}

\debugbox{
    \begin{minipage}{\linewidth}
        \centering
        \debugbox{
            \begin{minipage}{0.45\linewidth}
                \centering
                \includegraphics[width=\linewidth]{code/C++/DCCspec/data/images/spectra_real_consteps_20240826_143326_m914_init.png}        
            \end{minipage}
        }
        \debugbox{
            \begin{minipage}{0.45\linewidth}
                \centering
                \includegraphics[width=\linewidth]{code/C++/DCCspec/data/images/spectra_real_consteps_20240826_143326_m914_spec.png}        
            \end{minipage}
        }
    \end{minipage}
}

\debugbox{
    \begin{minipage}{\linewidth}
        \centering
        \debugbox{
            \begin{minipage}{0.45\linewidth}
                \centering
                \includegraphics[width=\linewidth]{code/C++/DCCspec/data/images/spectra_real_consteps_20240826_143341_m1000_init.png}        
            \end{minipage}
        }
        \debugbox{
            \begin{minipage}{0.45\linewidth}
                \centering
                \includegraphics[width=\linewidth]{code/C++/DCCspec/data/images/spectra_real_consteps_20240826_143341_m1000_spec.png}        
            \end{minipage}
        }
    \end{minipage}
}
With increasing mass, one observers increasing width of the spectrum, as well as a second dominant bump away from $p=0\,\text{GeV}$, namely roughly at the particle mass.