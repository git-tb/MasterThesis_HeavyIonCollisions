\section{Greens Functions of Klein-Gordon Equation}
\label{sec:Apdx_GreensFunctionsKG}
Solve $(\partial_\mu\partial^\mu-m^2)D(x)=\delta^{(4)}(x)$. Using $$D(x)=\int\frac{\dt^4p}{(2\pi)^4}\tilde{D}(p)e^{\imagu px}$$ one finds
\begin{subequations}
    \begin{align}
        \int\frac{\dt^4p}{(2\pi)^4}(-p^2-m^2)\tilde{D}(p)e^{\imagu px}&=\delta^{(4)}(x)\\
        \implies\qquad\tilde{D}(p)&=-\frac{1}{p^2+m^2}
    \end{align}
\end{subequations}
    and thus
    \begin{subequations}
        \begin{align}
            D(x)&=\int\frac{\dt^4p}{(2\pi)^4}\frac{-1}{p^2+m^2}e^{-\imagu px}\\
            &=-\int\frac{\dt^3p}{(2\pi)^3}\frac{\dt p^0}{2\pi}\frac{1}{(p^0+\omega_{\vec{p}})(-p^0+\omega_{\vec{p}})}e^{\imagu(p^0t-\vec{p}\vec{x})}
        \end{align}
    \end{subequations}
    The $p^0$-integration encounter 2 poles at ${\pm\omega_{\vec{p}}}$ on the real axis that need to be treated carefully. This is done by translating the integral over the real variable to an integral along a contour in the complex plane that approaches the real axis in a suitable limiting procedure. Different choices of how to choose this contour will lead to different boundary conditions. Consider for example the following contour:
\begin{defin}[Retarded Propagator, Contour]{def:RetPropCont}
        \centering
        \begin{tikzpicture}
            \draw[->] (0,-1) -- (0,1) node[above]{$\Im (p^0$)};
            \draw[->] (-3.5,0) -- (3.5,0) node[right]{$\Re (p^0$)};
    
            \coordinate[label=above:$-\omega_{\vec{p}}$] (n1) at (-2,0);
            \draw[fill] (n1) circle (0.05);
    
            \coordinate[label=above:$\omega_{\vec{p}}$] (n2) at (2,0);
            \draw[fill] (n2) circle (0.05);
    
            \draw[line width=1,color=black] (-2.5,0) arc (-180:0:0.5);
            \draw[line width=1,color=black,-{Latex[length=7,width=7,sep=2]}] (-3,0) -- (-2.5,0);
            \draw[line width=1,color=black] (-3,0) -- (-2.5,0);
            \draw[line width=1,color=black] (-1.5,0) -- (1.5,0);
            \draw[line width=1,color=black] (1.5,0) arc (-180:0:0.5);
            \draw[line width=1,color=black,-{Latex[length=7,width=7,sep=2]}] (2.5,0) -- (3,0);
            \draw[line width=1,color=black] (2.5,0) -- (3,0);
        \end{tikzpicture}
\end{defin}

    If $t>0$, close the integration contour in the upper imaginary half plane to achieve convergence. Making use of the residue theorem of complex analysis, one finds
    \begin{align}
        \int\frac{\dt p^0}{2\pi}\frac{1}{(p^0+\omega_{\vec{p}})(-p^0+\omega_{\vec{p}})}e^{\imagu p^0t}&=2\pi\imagu\Big(\lim_{p^0\to\omega_{\vec{p}}}\frac{1}{2\pi}(p^0-\omega_{\vec{p}})\frac{e^{\imagu p^0t}}{(p^0+\omega_{\vec{p}})(-p^0+\omega_{\vec{p}})}\nonumber\\
        &\phantom{=}+\lim_{p^0\to-\omega_{\vec{p}}}\frac{1}{2\pi}(p^0+\omega_{\vec{p}})\frac{e^{\imagu p^0t}}{(p^0+\omega_{\vec{p}})(-p^0+\omega_{\vec{p}})}\Big)\\
        &=\imagu\Big(\frac{e^{-\imagu\omega_{\vec{p}}t}-e^{\imagu\omega_{\vec{p}}t}}{2\omega_{\vec{p}}}\Big)\,.
    \end{align}
    For $t<0$, close the integration in the lower half plane, such that there is no residue within the integration contour. This leads to
    \begin{multline}
        D_R(x)=\frac{1}{\imagu}\Theta(t)\int\frac{\dt^3p}{(2\pi)^3}\frac{1}{2\omega_{\vec{p}}}(e^{-\imagu(\omega_{\vec{p}}t-\vec{p}\vec{x})}-e^{\imagu(\omega_{\vec{p}}t-\vec{p}\vec{x})})\\=-\imagu\Theta(x^0)\int\frac{\dt^3p}{(2\pi)^3}\frac{1}{2\omega_{\vec{p}}}(e^{\imagu px}-e^{-\imagu px})=\Theta(x^0)\Delta(x)\equiv\Delta_R(x)\,.
    \end{multline}

The same logic leads to the advanced propagator, defined by the following integration contour:
\begin{defin}[Advanced Propagator, Contour]{def:AdvPropCont}
    \centering
    \begin{tikzpicture}
        \draw[->] (0,-1) -- (0,1) node[above]{$\Im (p^0$)};
        \draw[->] (-3.5,0) -- (3.5,0) node[right]{$\Re (p^0$)};

        \coordinate[label=below:$-\omega_{\vec{p}}$] (n1) at (-2,0);
        \draw[fill] (n1) circle (0.05);

        \coordinate[label=below:$\omega_{\vec{p}}$] (n2) at (2,0);
        \draw[fill] (n2) circle (0.05);

        \draw[line width=1,color=black] (-2.5,0) arc (180:0:0.5);
        \draw[line width=1,color=black,-{Latex[length=7,width=7,sep=2]}] (-3,0) -- (-2.5,0);
        \draw[line width=1,color=black] (-3,0) -- (-2.5,0);
        \draw[line width=1,color=black] (-1.5,0) -- (1.5,0);
        \draw[line width=1,color=black] (1.5,0) arc (180:0:0.5);
        \draw[line width=1,color=black,-{Latex[length=7,width=7,sep=2]}] (2.5,0) -- (3,0);
        \draw[line width=1,color=black] (2.5,0) -- (3,0);
    \end{tikzpicture}
\end{defin}