\documentclass[letterpaper,twocolumn,10pt]{article} 

\usepackage[dvipsnames]{xcolor}
\usepackage{annotate-equations}
\renewcommand{\eqnannotationtext}[1]{#1}
\renewcommand{\eqnhighlightheight}{}

\usepackage{amsmath}
\usepackage{amsthm}
\usepackage{amssymb}
\usepackage{mathtools, nccmath}
\usepackage{wrapfig}
\usepackage{comment}

% To generate dummy text
\usepackage{blindtext}


%color
%\usepackage[dvipsnames]{xcolor}
% \usepackage{xcolor}


%\usepackage[pdftex]{graphicx}
\usepackage{graphicx}
% declare the path(s) for graphic files
%\graphicspath{{../Figures/}}

% extensions so you won't have to specify these with
% every instance of \includegraphics
% \DeclareGraphicsExtensions{.pdf,.jpeg,.png}

% for custom commands
\usepackage{xspace}

% table alignment
\usepackage{array}
\usepackage{ragged2e}
\newcolumntype{P}[1]{>{\RaggedRight\hspace{0pt}}p{#1}}
\newcolumntype{X}[1]{>{\RaggedRight\hspace*{0pt}}p{#1}}

% color box
%\usepackage{tcolorbox}


% for tikz
%\usepackage{tikz}
%%\usetikzlibrary{trees}
%\usetikzlibrary{arrows,shapes,positioning,shadows,trees,mindmap}
%% \usepackage{forest}
%\usepackage[edges]{forest}
%\usetikzlibrary{arrows.meta}
\colorlet{linecol}{black!75}
\usepackage{xkcdcolors} % xkcd colors


% for colorful equation
%\usepackage{tikz}
%\usetikzlibrary{backgrounds}
%\usetikzlibrary{arrows,shapes}
%\usetikzlibrary{tikzmark}
%\usetikzlibrary{calc}
%% Commands for Highlighting text -- non tikz method
%\newcommand{\highlight}[2]{\colorbox{#1!17}{$\displaystyle #2$}}
%%\newcommand{\highlight}[2]{\colorbox{#1!17}{$#2$}}
%\newcommand{\highlightdark}[2]{\colorbox{#1!47}{$\displaystyle #2$}}

% my custom colors for shading
\colorlet{mhpurple}{Plum!80}


%% Commands for Highlighting text -- non tikz method
%\renewcommand{\highlight}[2]{\colorbox{#1!17}{#2}}
%\renewcommand{\highlightdark}[2]{\colorbox{#1!47}{#2}}

% Some math definitions
\newcommand{\lap}{\mathrm{Lap}}
\newcommand{\pr}{\mathrm{Pr}}

\newcommand{\Tset}{\mathcal{T}}
\newcommand{\Dset}{\mathcal{D}}
\newcommand{\Rbound}{\widetilde{\mathcal{R}}}

\begin{document}

\title{Annotated Equations in Latex Using Tikz}

\author{
{\bf Sibin Mohan, adapted to annotate\_equations by ST John}\\
{https://github.com/st--}
% copy the following lines to add more authors
% \and
% {\rm Name}\\
%Name Institution
} % end author

\maketitle

% Start with Two-Column Examples
\blindtext
\vspace{2\baselineskip}

%%%%%
\begin{equation}
    \vspace{\baselineskip}
            \label{eq:epsilon}
                \pr[\eqnmarkbox[red]{x}{\mathcal{X}(\cdot)}\in \eqnmarkbox[blue]{s}{\mathcal{S}}] \leq e^\epsilon \cdot \pr[\eqnmark{x2}{\mathcal{X}}(\cdot)\in \mathcal{S}]
\end{equation}
\annotate[yshift=2em,xshift=2ex]{above,left}{x,x2}{\textbf{system state}}
\annotate[yshift=-1.5em]{below,right}{s}{\textbf{$\mathcal{S} \subseteq \mathrm{Range}(\mathcal{X})$}}
%%%%%

% More complex annotations
\blindtext
\blindtext
%%%%%
\begin{figure}[htb]
    \vspace{2\baselineskip}
    \begin{equation}
        \pr[\mathcal{R}(\eqnmarkbox[red]{ts}{\tau_i},\eqnmarkbox[blue]{js}{j})\in \mathcal{S}] \leq e^\epsilon \pr[\mathcal{R}(\eqnmarkbox[red]{td}{\tau_i'},\eqnmarkbox[blue]{jd}{j'})\in \mathcal{S}]
    \label{eq:dp_one_instance}
    \end{equation}
    % TODO arrow color = mark color!57, annotation color = mark color!67
    \annotatetwo[yshift=2em]{above}{ts}{td}{\textbf{$\tau_i,\tau' \in \Gamma$}, \textbf{the set of Tasks}}
    \annotatetwo[yshift=-2em]{below, label below}{js}{jd}{\textbf{$j,j'\in \mathbb{N}$}}
    \vspace{\baselineskip}
    \caption{A More Complex Example for Annotated Equations, this time inside a figure construct.}
\end{figure}
%%%%%

% Single column with more complex examples
\onecolumn

% Wrapping text around equations
\blindtext
%%%%%
\begin{wrapfigure}{l}{0.5\columnwidth}
    \vspace{\baselineskip}
    \begin{equation}
    \label{eq:laplace_density}
        \lap (x\ |\ \eqnmarkbox[red]{u}{\mu}, \eqnmarkbox[blue]{b}{b}) = \frac{1}{2b} \mathrm{exp}(-\frac{|x-\mu|}{b}) 
    \end{equation}
    \annotate[yshift=2em]{above,right}{u}{\textbf{location parameter, mean}}
    \annotate[yshift=-1.5em]{below,right}{b}{\textbf{$b >0$, scale parameter}}
    \vspace{0.5\baselineskip}
    \caption{An example in the single column format using the wrapfig construct.}
    \vspace{0.5\baselineskip}
\end{wrapfigure}
%%%%%

% Annotated Equations Side-by-Side
\blindtext
%%%%%
\begin{figure}[h]
    \vspace{\baselineskip}
\begin{minipage}{0.5\columnwidth}
\begin{equation*}
\label{eq:ab_flushing}
N_{i} = 
\color{purple}
\overbrace{ 
    \eqnmarkbox[purple]{qp}{\color{black} Q_p} \color{black}\big( \eqnmarkbox[NavyBlue]{tj1}{\color{black} t_{j+1}} \color{black} - \eqnmarkbox[Bittersweet]{tj}{\color{black} t_{j}}  
    \color{black}\big) |   
    \eqnmarkbox[purple]{aj}{\color{black} \forall j}
    \color{black}\big)
}^{\substack{\text{\sf \footnotesize \textcolor{purple!85}{Some annotation about the 
	}} \\ \text{\sf \footnotesize \textcolor{purple!85}{entire equation here.
}} } }
\end{equation*}
\vspace*{0.5\baselineskip}
\annotate[yshift=-1.8em]{below,left}{tj1}{\textsf{\footnotesize property of (j+1)\textsuperscript{th} item}}
\annotate[yshift=-1.8em]{below,right}{tj}{\textsf{\footnotesize j\textsuperscript{th} item}}
\end{minipage}
\hfil
\begin{minipage}{0.5\columnwidth}
\begin{equation*}
    \label{eq:ab_crypto}
    \hspace*{-6em}
    X_{i} = \frac{1}{\sum_{i=1}^{\eqnmarkbox[purple]{n}{N}} 
    \sum_{j=1}^{\eqnmarkbox[NavyBlue]{mi}{M_i}} 
    \tfrac{\eqnmarkbox[Bittersweet]{lij}{l_i^j}}{\eqnmarkbox[OliveGreen]{lmax}{l^{max}}}
    }
\end{equation*}
\vspace*{0.8\baselineskip}
\annotate[yshift=1.8em,color=Plum]{above,left}{n}{\textsf{\footnotesize number of objects}} %%% TODO change arrow color too!
\annotate[yshift=3.5em]{above,right}{mi}{\textsf{\footnotesize number of other objects}}
\annotate[yshift=1.9em]{above,right}{lij}{\textsf{\footnotesize size of j\textsuperscript{th} service}}
\annotate[yshift=-1.5em,color=xkcdHunterGreen]{below,left}{lmax}{\textsf{\footnotesize maximum obj size}}
\end{minipage}
\caption{Two Equations side-by-side using minipage and figure constructs.}
\end{figure}
%%%%%

\blindtext

\end{document}
