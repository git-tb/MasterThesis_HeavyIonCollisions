\documentclass[xcolor={dvipsnames,rgb},aspectratio=1610]{beamer}
% xcolor: e.g. provides \colorlet
% dvipsnames option: extra named colours
% aspectratio=1610: for modern widescreen monitors...
% see https://www.overleaf.com/learn/latex/Using_colours_in_LaTeX for more detail

% equation annotations also work in articles, of course (then might need explicit \usepackage[dvipsnames]{xcolor}

%%% standard math packages for equations:
\usepackage{amsmath}
\usepackage{amssymb}
\usepackage{mathtools}

\usepackage{annotate-equations}

\begin{document}

\begin{frame}{Runtime complexity}
    \LARGE %%% increase font size to make equation more readable!

    %%% we define our own colors upfront - this makes it easier to keep it consistent if you change your mind
    \colorlet{colorp}{NavyBlue}
    \colorlet{colorT}{WildStrawberry}
    \colorlet{colork}{OliveGreen}
    \colorlet{colorM}{RoyalPurple}
    \colorlet{colorNb}{Plum}
    \colorlet{colorIs}{black}

    \begin{equation*}
        \color{BurntOrange}
        \mathcal{O}\big(
            (
            % \tikzmarknode is what links parts of the equation and corresponding annotations
            \eqnmarkbox[colorp]{p1}{p}
            \eqnmarkbox[colork]{k1}{\kappa}^3 % note that we have the ^3 outside the \tikzmarknode
            )
            \eqnmarkbox{T1}{T}
            +
            (
            \eqnmarkbox[colorp]{p2}{p} % tikzmarks need distinct names!
            \eqnmark[colork]{k2}{\kappa}
            )
            (
            \eqnmarkbox[colorT]{T2}{T}^2
            \tikzmarknode{Is}{|\mathcal{I}^*|}
            \eqnmarkbox[colorNb]{Nb}{N}_{\!\!\eqnmarkbox[BurntOrange]{b}{b}}
            \eqnmark[colorM]{M}{M}
            )
        \big)
    \end{equation*}
    \annotatetwo[yshift=1em]{above}{p1}{p2}{\# of nodes}
    \annotatetwo[yshift=-1em,xshift=0.2ex]{below}{T1}{T2}{\# of graphs in $\hat{\mathcal{G}}_T$}
    \annotatetwo[yshift=-2em]{below}{k1}{k2}{max.\ indegree in $\hat{\mathcal{G}}_T$}
    \annotate[yshift=3em]{above,left}{Is}{size of set of allowed interventions}
    \annotate[yshift=1em]{above}{Nb}{\# of samples per batch}
    \annotate[yshift=-1em]{below}{M}{\# of samples for $\mathbb{E}_y$}
    \annotate[yshift=-3em]{below,left}{b}{batch index}
\end{frame}

\end{document}
